\section{Progettazione logica}

\subsection*{Volume dei dati}
%
% Questa sezione serve ad illustrare qual è il carico che la base di dati
% dovrà sopportare.
% A tal fine, è necessario prevedere un volume di dati attesi.
% Compilare la tabella sottostante, per ciascun concetto identificato nello
% schema E-R. I volumi devono essere stimati dallo studente in maniera
% ragionevole rispetto all’operatività presunta dell’applicativo.
%
\begin{tabularx}{\linewidth}{|X|l|X|}
    \hline
    \rowcolor{tblhdrcolor}
    \multicolumn{1}{|c|}{\textbf{Concetto nello schema}}
     & \multicolumn{1}{|c|}{\textbf{Tipo}
        \footnote{Indicare con E le entità, con R le relazioni}}
     & \multicolumn{1}{|c|}{\textbf{Volume atteso}}
    \\\hline
    Cinema
     & E
     & 10
    \\\hline
    Sala
     & E
     & 50 (in media 5 sale per ogni cinema)
    \\ \hline
    Posto
     & E
     & 10 000 (in media 200 posti per ogni sala)
    \\ \hline
    Dipendente
     & E
     & 200
    \\ \hline
    Maschera
     & E
     & 50
    \\ \hline
    Proiezionista
     & E
     & 150
    \\ \hline
    Turno
     & E
     & 1 200 (in media un turno per ogni dipendente per sei giorni a settimana)
    \\ \hline
    Film
     & E
     & 600 000
    \\ \hline
    Proiezione
     & E
     & 1 400 (in media 4 proiezioni al giorno per ogni sala per una settimana
    prima del cambio del palinsesto)
    \\ \hline
    Cliente
     & E
     & 100 000
    \\ \hline
    Prenotazione
     & E
     & 560 000 (in media la prenotazione del 50\% dei posti disponibili al
    giorno in un mese)
    \\ \hline
    Prenotazione in attesa
     & E
     & 28 000 (in media il 5\% delle prenotazioni)
    \\ \hline
    Prenotazione confermata
     & E
     & 532 000 (in media il 95\% delle prenotazioni)
    \\ \hline
    Prenotazione validata
     & E
     & 425 600 (in media l'80\% delle prenotazioni confermate)
    \\ \hline
    Prenotazione scaduta
     & E
     & 53 200 (in media il 10\% delle prenotazioni confermate)
    \\ \hline
    Prenotazione annullata
     & E
     & 53 200 (in media il 10\% delle prenotazioni confermate)
    \\ \hline
    CinemaSala
     & R
     & 50
    \\ \hline
    SalaPosto
     & R
     & 10 000
    \\ \hline
    ProiezioneSala
     & R
     & 1 400
    \\ \hline
    PostoPrenotazione
     & R
     & 560 000
    \\ \hline
    PrenotazioneProiezione
     & R
     & 560 000
    \\ \hline
    ClientePrenotazione
     & R
     & 560 000
    \\ \hline
    FilmProiezione
     & R
     & 14 000
    \\ \hline
    ProiezioneProiezionista
     & R
     & 14 000
    \\ \hline
    Validazione
     & R
     & 425 600
    \\ \hline
    DipendenteTurno
     & R
     & 1 200
    \\ \hline
    CinemaTurno
     & R
     & 1 200
    \\ \hline
\end{tabularx}

\subsection*{Tavola delle operazioni}
%
% Rappresentare nella tabella sottostante tutte le operazioni sulla base
% di dati che devono essere supportate dall’applicazione, con la
% frequenza attesa.
% Le operazioni da supportare devono essere desunte dalle specifiche raccolte.
%
\begin{tabularx}{\linewidth}{|l|X|X|}
    \hline
    \rowcolor{tblhdrcolor}
    \multicolumn{1}{|c|}{\textbf{Cod.}}
     & \multicolumn{1}{|c|}{\textbf{Descrizione}}
     & \multicolumn{1}{|c|}{\textbf{Frequenza attesa}}
    \\ \hline
    Op. 1
     & Mostra tutti i cinema e i loro relativi dati.
     & 5000/giorno
    \\ \hline
    Op. 2
     & Inserisci un nuovo cinema.
     & 1/anno
    \\ \hline
    Op. 3
     & Rimuovi un cinema.
     & 1/lustro
    \\ \hline
    Op. 4
     & Mostra le sale di un cinema e i loro relativi dati.
     & 1/anno
    \\ \hline
    Op. 5
     & Inserisci una nuova sala.
     & 1/anno
    \\ \hline
    Op. 6
     & Rimuovi una sala.
     & 1/lustro
    \\ \hline
    Op. 7
     & Mostra tutte le proiezioni in programma ed i dati relativi al
    proiezionista assegnato.
     & 40/mese
    \\ \hline
    Op. 8
     & Inserisci una nuova proiezione in programma.
     & 5600/mese
    \\ \hline
    Op. 9
     & Elimina una proiezione in programma.
     & 1/mese
    \\ \hline
    Op. 10
     & Assegna un proiezionista ad una proiezione in programma.
     & 1400/settimana
    \\ \hline
    Op. 11
     & Mostra tutte le proiezioni in programma in un cinema e le informazioni
    sui relativi film proiettati.
     & 5000/giorno
    \\ \hline
    Op. 12
     & Mostra tutti i posti disponibili per una proiezione.
     & 5000/giorno
    \\ \hline
    Op. 13
     & Effettua un pagamento ed inserisci una nuova prenotazione.
     & 5000/giorno
    \\ \hline
    Op. 14
     & Effettua un rimborso ed annulla una prenotazione.
     & 500/giorno
    \\ \hline
    Op. 15
     & Valida una prenotazione.
     & 4500/giorno
    \\ \hline
    Op. 16
     & Mostra i dipendenti e i loro relativi dati.
     & 50/settimana
    \\ \hline
    Op. 17
     & Inserisci un nuovo dipendente.
     & 10/anno
    \\ \hline
    Op. 18
     & Rimuovi un dipendente.
     & 10/anno
    \\ \hline
    Op. 19
     & Mostra i turni in programma.
     & 50/settimana
    \\ \hline
    Op. 20
     & Inserisci un nuovo turno assegnato ad un dipendente.
     & 50/settimana
    \\ \hline
    Op. 21
     & Elimina un turno assegnato ad un dipendente.
     & 50/settimana
    \\ \hline
    Op. 22
     & Mostra tutte le proiezioni che non sono associate ad un proiezionista.
     & 1/giorno
    \\ \hline
    Op. 23
     & Mostra tutti i periodi temporali in cui un cinema è sprovvisto di
    almeno due maschere durante la propria fascia oraria d'apertura.
     & 1/giorno
    \\ \hline
    Op. 24
     & Mostra per ogni cinema e per ogni sala il numero delle prenotazioni
    confermate, il numero delle prenotazioni annullate ed il numero delle
    prenotazioni scadute.
     & 1/giorno
    \\ \hline
\end{tabularx}

\pagebreak
\subsection*{Costo delle operazioni}
%
% In riferimento a tutte le operazioni precedentemente indicate, calcolarne
% il costo supponendo, per questa fase del progetto, che il costo in
% scrittura di un dato sia doppio rispetto a quello in lettura.
%

\begin{tabularx}{\linewidth}{|X|X|X|X|}
    \hline
    \multicolumn{4}{|>{\columncolor{tblhdrcolor}}c|}{
    \textbf{Operazione 1}}                      \\\hline
    \rowcolor{tblhdrcolor}
    \multicolumn{1}{|c|}{\textbf{Concetto}}
     & \multicolumn{1}{|c|}{\textbf{Costrutto}}
     & \multicolumn{1}{|c|}{\textbf{Accessi}}
     & \multicolumn{1}{|c|}{\textbf{Tipo}}
    \\ \hline
    Cinema
     & E
     & 1
     & L
    \\\hline
    \multicolumn{4}{|>{\columncolor{tblhdrcolor}}l|}{
    Costo: 5 000/giorno}                        \\\hline
\end{tabularx}

\begin{tabularx}{\linewidth}{|X|X|X|X|}
    \hline
    \multicolumn{4}{|>{\columncolor{tblhdrcolor}}c|}{
    \textbf{Operazione 2}}                      \\\hline
    \rowcolor{tblhdrcolor}
    \multicolumn{1}{|c|}{\textbf{Concetto}}
     & \multicolumn{1}{|c|}{\textbf{Costrutto}}
     & \multicolumn{1}{|c|}{\textbf{Accessi}}
     & \multicolumn{1}{|c|}{\textbf{Tipo}}
    \\ \hline
    Cinema
     & E
     & 1
     & S
    \\\hline
    \multicolumn{4}{|>{\columncolor{tblhdrcolor}}l|}{
    Costo: 2/anno}                              \\\hline
\end{tabularx}

\begin{tabularx}{\linewidth}{|X|X|X|X|}
    \hline
    \multicolumn{4}{|>{\columncolor{tblhdrcolor}}c|}{
    \textbf{Operazione 3}}                      \\\hline
    \rowcolor{tblhdrcolor}
    \multicolumn{1}{|c|}{\textbf{Concetto}}
     & \multicolumn{1}{|c|}{\textbf{Costrutto}}
     & \multicolumn{1}{|c|}{\textbf{Accessi}}
     & \multicolumn{1}{|c|}{\textbf{Tipo}}
    \\ \hline
    Cinema
     & E
     & 1
     & S
    \\\hline
    \multicolumn{4}{|>{\columncolor{tblhdrcolor}}l|}{
    Costo: 2/lustro}                            \\\hline
\end{tabularx}

\begin{tabularx}{\linewidth}{|X|X|X|X|}
    \hline
    \multicolumn{4}{|>{\columncolor{tblhdrcolor}}c|}{
    \textbf{Operazione 4}}                      \\\hline
    \rowcolor{tblhdrcolor}
    \multicolumn{1}{|c|}{\textbf{Concetto}}
     & \multicolumn{1}{|c|}{\textbf{Costrutto}}
     & \multicolumn{1}{|c|}{\textbf{Accessi}}
     & \multicolumn{1}{|c|}{\textbf{Tipo}}
    \\ \hline
    Cinema
     & E
     & 1
     & L
    \\\hline
    Sala
     & E
     & 1
     & L
    \\\hline
    Posto
     & E
     & 1
     & L
    \\\hline
    \multicolumn{4}{|>{\columncolor{tblhdrcolor}}l|}{
    Costo: 3/anno}                              \\\hline
\end{tabularx}

\begin{tabularx}{\linewidth}{|X|X|X|X|}
    \hline
    \multicolumn{4}{|>{\columncolor{tblhdrcolor}}c|}{
    \textbf{Operazione 5}}                      \\\hline
    \rowcolor{tblhdrcolor}
    \multicolumn{1}{|c|}{\textbf{Concetto}}
     & \multicolumn{1}{|c|}{\textbf{Costrutto}}
     & \multicolumn{1}{|c|}{\textbf{Accessi}}
     & \multicolumn{1}{|c|}{\textbf{Tipo}}
    \\ \hline
    Sala
     & E
     & 1
     & S
    \\\hline
    Posto
     & E
     & 1
     & S
    \\\hline
    \multicolumn{4}{|>{\columncolor{tblhdrcolor}}l|}{
    Costo: 4/anno}                              \\\hline
\end{tabularx}

\begin{tabularx}{\linewidth}{|X|X|X|X|}
    \hline
    \multicolumn{4}{|>{\columncolor{tblhdrcolor}}c|}{
    \textbf{Operazione 6}}                      \\\hline
    \rowcolor{tblhdrcolor}
    \multicolumn{1}{|c|}{\textbf{Concetto}}
     & \multicolumn{1}{|c|}{\textbf{Costrutto}}
     & \multicolumn{1}{|c|}{\textbf{Accessi}}
     & \multicolumn{1}{|c|}{\textbf{Tipo}}
    \\ \hline
    Sala
     & E
     & 1
     & S
    \\\hline
    \multicolumn{4}{|>{\columncolor{tblhdrcolor}}l|}{
    Costo: 2/lustro}                            \\\hline
\end{tabularx}

\begin{tabularx}{\linewidth}{|X|X|X|X|}
    \hline
    \multicolumn{4}{|>{\columncolor{tblhdrcolor}}c|}{
    \textbf{Operazione 7}}                      \\\hline
    \rowcolor{tblhdrcolor}
    \multicolumn{1}{|c|}{\textbf{Concetto}}
     & \multicolumn{1}{|c|}{\textbf{Costrutto}}
     & \multicolumn{1}{|c|}{\textbf{Accessi}}
     & \multicolumn{1}{|c|}{\textbf{Tipo}}
    \\ \hline
    Proiezione
     & E
     & 1
     & L
    \\\hline
    Film
     & E
     & 1
     & L
    \\\hline
    \multicolumn{4}{|>{\columncolor{tblhdrcolor}}l|}{
    Costo: 80/mese}                             \\\hline
\end{tabularx}

\begin{tabularx}{\linewidth}{|X|X|X|X|}
    \hline
    \multicolumn{4}{|>{\columncolor{tblhdrcolor}}c|}{
    \textbf{Operazione 8}}                      \\\hline
    \rowcolor{tblhdrcolor}
    \multicolumn{1}{|c|}{\textbf{Concetto}}
     & \multicolumn{1}{|c|}{\textbf{Costrutto}}
     & \multicolumn{1}{|c|}{\textbf{Accessi}}
     & \multicolumn{1}{|c|}{\textbf{Tipo}}
    \\ \hline
    Proiezione
     & E
     & 1
     & S
    \\\hline
    \multicolumn{4}{|>{\columncolor{tblhdrcolor}}l|}{
    Costo: 11 200/mese}                         \\\hline
\end{tabularx}

\begin{tabularx}{\linewidth}{|X|X|X|X|}
    \hline
    \multicolumn{4}{|>{\columncolor{tblhdrcolor}}c|}{
    \textbf{Operazione 9}}                      \\\hline
    \rowcolor{tblhdrcolor}
    \multicolumn{1}{|c|}{\textbf{Concetto}}
     & \multicolumn{1}{|c|}{\textbf{Costrutto}}
     & \multicolumn{1}{|c|}{\textbf{Accessi}}
     & \multicolumn{1}{|c|}{\textbf{Tipo}}
    \\ \hline
    Proiezione
     & E
     & 1
     & S
    \\\hline
    \multicolumn{4}{|>{\columncolor{tblhdrcolor}}l|}{
    Costo: 2/mese}                              \\\hline
\end{tabularx}

\begin{tabularx}{\linewidth}{|l|X|X|X|}
    \hline
    \multicolumn{4}{|>{\columncolor{tblhdrcolor}}c|}{
    \textbf{Operazione 10}}                     \\\hline
    \rowcolor{tblhdrcolor}
    \multicolumn{1}{|c|}{\textbf{Concetto}}
     & \multicolumn{1}{|c|}{\textbf{Costrutto}}
     & \multicolumn{1}{|c|}{\textbf{Accessi}}
     & \multicolumn{1}{|c|}{\textbf{Tipo}}
    \\ \hline
    Proiezione
     & E
     & 1
     & S
    \\\hline
    \multicolumn{4}{|>{\columncolor{tblhdrcolor}}l|}{
    Costo: 2 800/settimana}                     \\\hline
\end{tabularx}

\begin{tabularx}{\linewidth}{|X|X|X|X|}
    \hline
    \multicolumn{4}{|>{\columncolor{tblhdrcolor}}c|}{
    \textbf{Operazione 11}}                     \\\hline
    \rowcolor{tblhdrcolor}
    \multicolumn{1}{|c|}{\textbf{Concetto}}
     & \multicolumn{1}{|c|}{\textbf{Costrutto}}
     & \multicolumn{1}{|c|}{\textbf{Accessi}}
     & \multicolumn{1}{|c|}{\textbf{Tipo}}
    \\ \hline
    Proiezione
     & E
     & 1
     & L
    \\\hline
    Film
     & E
     & 1
     & L
    \\\hline
    \multicolumn{4}{|>{\columncolor{tblhdrcolor}}l|}{
    Costo: 10 000/giorno}                       \\\hline
\end{tabularx}

\begin{tabularx}{\linewidth}{|l|X|X|X|}
    \hline
    \multicolumn{4}{|>{\columncolor{tblhdrcolor}}c|}{
    \textbf{Operazione 12}}                     \\\hline
    \rowcolor{tblhdrcolor}
    \multicolumn{1}{|c|}{\textbf{Concetto}}
     & \multicolumn{1}{|c|}{\textbf{Costrutto}}
     & \multicolumn{1}{|c|}{\textbf{Accessi}}
     & \multicolumn{1}{|c|}{\textbf{Tipo}}
    \\ \hline
    Prenotazione in attesa
     & E
     & 1
     & L
    \\\hline
    Prenotazione confermata
     & E
     & 1
     & L
    \\\hline
    Posto
     & E
     & 1
     & L
    \\\hline
    \multicolumn{4}{|>{\columncolor{tblhdrcolor}}l|}{
    Costo: 15 000/giorno}                       \\\hline
\end{tabularx}

\begin{tabularx}{\linewidth}{|l|X|X|X|}
    \hline
    \multicolumn{4}{|>{\columncolor{tblhdrcolor}}c|}{
    \textbf{Operazione 13}}                     \\\hline
    \rowcolor{tblhdrcolor}
    \multicolumn{1}{|c|}{\textbf{Concetto}}
     & \multicolumn{1}{|c|}{\textbf{Costrutto}}
     & \multicolumn{1}{|c|}{\textbf{Accessi}}
     & \multicolumn{1}{|c|}{\textbf{Tipo}}
    \\ \hline
    Prenotazione confermata
     & E
     & 1
     & S
    \\\hline
    Prenotazione in attesa
     & E
     & 1
     & S
    \\\hline
    \multicolumn{4}{|>{\columncolor{tblhdrcolor}}l|}{
    Costo: 20 000/giorno}                       \\\hline
\end{tabularx}

\begin{tabularx}{\linewidth}{|l|X|X|X|}
    \hline
    \multicolumn{4}{|>{\columncolor{tblhdrcolor}}c|}{
    \textbf{Operazione 14}}                     \\\hline
    \rowcolor{tblhdrcolor}
    \multicolumn{1}{|c|}{\textbf{Concetto}}
     & \multicolumn{1}{|c|}{\textbf{Costrutto}}
     & \multicolumn{1}{|c|}{\textbf{Accessi}}
     & \multicolumn{1}{|c|}{\textbf{Tipo}}
    \\ \hline
    Prenotazione confermata
     & E
     & 1
     & S
    \\ \hline
    Prenotazione annullata
     & E
     & 1
     & S
    \\\hline
    \multicolumn{4}{|>{\columncolor{tblhdrcolor}}l|}{
    Costo: 2 000/giorno}                        \\\hline
\end{tabularx}

\begin{tabularx}{\linewidth}{|l|X|X|X|}
    \hline
    \multicolumn{4}{|>{\columncolor{tblhdrcolor}}c|}{
    \textbf{Operazione 15}}                     \\\hline
    \rowcolor{tblhdrcolor}
    \multicolumn{1}{|c|}{\textbf{Concetto}}
     & \multicolumn{1}{|c|}{\textbf{Costrutto}}
     & \multicolumn{1}{|c|}{\textbf{Accessi}}
     & \multicolumn{1}{|c|}{\textbf{Tipo}}
    \\ \hline
    Prenotazione confermata
     & E
     & 1
     & S
    \\ \hline
    Prenotazione validata
     & E
     & 1
     & S
    \\\hline
    \multicolumn{4}{|>{\columncolor{tblhdrcolor}}l|}{
    Costo: 18 000/giorno}                       \\\hline
\end{tabularx}

\begin{tabularx}{\linewidth}{|X|X|X|X|}
    \hline
    \multicolumn{4}{|>{\columncolor{tblhdrcolor}}c|}{
    \textbf{Operazione 16}}                     \\\hline
    \rowcolor{tblhdrcolor}
    \multicolumn{1}{|c|}{\textbf{Concetto}}
     & \multicolumn{1}{|c|}{\textbf{Costrutto}}
     & \multicolumn{1}{|c|}{\textbf{Accessi}}
     & \multicolumn{1}{|c|}{\textbf{Tipo}}
    \\ \hline
    Dipendente
     & E
     & 1
     & L
    \\ \hline
    \multicolumn{4}{|>{\columncolor{tblhdrcolor}}l|}{
    Costo: 50/settimana}                        \\\hline
\end{tabularx}

\begin{tabularx}{\linewidth}{|X|X|X|X|}
    \hline
    \multicolumn{4}{|>{\columncolor{tblhdrcolor}}c|}{
    \textbf{Operazione 17}}                     \\\hline
    \rowcolor{tblhdrcolor}
    \multicolumn{1}{|c|}{\textbf{Concetto}}
     & \multicolumn{1}{|c|}{\textbf{Costrutto}}
     & \multicolumn{1}{|c|}{\textbf{Accessi}}
     & \multicolumn{1}{|c|}{\textbf{Tipo}}
    \\ \hline
    Dipendente
     & E
     & 1
     & S
    \\ \hline
    \multicolumn{4}{|>{\columncolor{tblhdrcolor}}l|}{
    Costo: 20/anno}                             \\\hline
\end{tabularx}

\begin{tabularx}{\linewidth}{|X|X|X|X|}
    \hline
    \multicolumn{4}{|>{\columncolor{tblhdrcolor}}c|}{
    \textbf{Operazione 18}}                     \\\hline
    \rowcolor{tblhdrcolor}
    \multicolumn{1}{|c|}{\textbf{Concetto}}
     & \multicolumn{1}{|c|}{\textbf{Costrutto}}
     & \multicolumn{1}{|c|}{\textbf{Accessi}}
     & \multicolumn{1}{|c|}{\textbf{Tipo}}
    \\ \hline
    Dipendente
     & E
     & 1
     & S
    \\ \hline
    \multicolumn{4}{|>{\columncolor{tblhdrcolor}}l|}{
    Costo: 20/anno}                             \\\hline
\end{tabularx}

\begin{tabularx}{\linewidth}{|l|X|X|X|}
    \hline
    \multicolumn{4}{|>{\columncolor{tblhdrcolor}}c|}{
    \textbf{Operazione 19}}                     \\\hline
    \rowcolor{tblhdrcolor}
    \multicolumn{1}{|c|}{\textbf{Concetto}}
     & \multicolumn{1}{|c|}{\textbf{Costrutto}}
     & \multicolumn{1}{|c|}{\textbf{Accessi}}
     & \multicolumn{1}{|c|}{\textbf{Tipo}}
    \\ \hline
    Turno
     & E
     & 1
     & L
    \\\hline
    Dipendenti
     & E
     & 1
     & L
    \\\hline
    \multicolumn{4}{|>{\columncolor{tblhdrcolor}}l|}{
    Costo: 100/settimana}                        \\\hline
\end{tabularx}

\begin{tabularx}{\linewidth}{|X|X|X|X|}
    \hline
    \multicolumn{4}{|>{\columncolor{tblhdrcolor}}c|}{
    \textbf{Operazione 20}}                     \\\hline
    \rowcolor{tblhdrcolor}
    \multicolumn{1}{|c|}{\textbf{Concetto}}
     & \multicolumn{1}{|c|}{\textbf{Costrutto}}
     & \multicolumn{1}{|c|}{\textbf{Accessi}}
     & \multicolumn{1}{|c|}{\textbf{Tipo}}
    \\ \hline
    Turno
     & E
     & 1
     & S
    \\\hline
    \multicolumn{4}{|>{\columncolor{tblhdrcolor}}l|}{
    Costo: 100/settimana}                       \\\hline
\end{tabularx}

\begin{tabularx}{\linewidth}{|X|X|X|X|}
    \hline
    \multicolumn{4}{|>{\columncolor{tblhdrcolor}}c|}{
    \textbf{Operazione 21}}                     \\\hline
    \rowcolor{tblhdrcolor}
    \multicolumn{1}{|c|}{\textbf{Concetto}}
     & \multicolumn{1}{|c|}{\textbf{Costrutto}}
     & \multicolumn{1}{|c|}{\textbf{Accessi}}
     & \multicolumn{1}{|c|}{\textbf{Tipo}}
    \\ \hline
    Turno
     & E
     & 1
     & S
    \\\hline
    \multicolumn{4}{|>{\columncolor{tblhdrcolor}}l|}{
    Costo: 100/settimana}                       \\\hline
\end{tabularx}

\begin{tabularx}{\linewidth}{|l|X|X|X|}
    \hline
    \multicolumn{4}{|>{\columncolor{tblhdrcolor}}c|}{
    \textbf{Operazione 22}}                     \\\hline
    \rowcolor{tblhdrcolor}
    \multicolumn{1}{|c|}{\textbf{Concetto}}
     & \multicolumn{1}{|c|}{\textbf{Costrutto}}
     & \multicolumn{1}{|c|}{\textbf{Accessi}}
     & \multicolumn{1}{|c|}{\textbf{Tipo}}
    \\ \hline
    Proiezione
     & E
     & 1
     & L
    \\\hline
    \multicolumn{4}{|>{\columncolor{tblhdrcolor}}l|}{
    Costo: 1/giorno}                            \\\hline
\end{tabularx}

\begin{tabularx}{\linewidth}{|X|X|X|X|}
    \hline
    \multicolumn{4}{|>{\columncolor{tblhdrcolor}}c|}{
    \textbf{Operazione 23}}                     \\\hline
    \rowcolor{tblhdrcolor}
    \multicolumn{1}{|c|}{\textbf{Concetto}}
     & \multicolumn{1}{|c|}{\textbf{Costrutto}}
     & \multicolumn{1}{|c|}{\textbf{Accessi}}
     & \multicolumn{1}{|c|}{\textbf{Tipo}}
    \\ \hline
    Cinema
     & E
     & 3
     & L
    \\\hline
    Turno
     & E
     & 2
     & L
    \\\hline
    Maschera
     & E
     & 2
     & L
    \\\hline
    \multicolumn{4}{|>{\columncolor{tblhdrcolor}}l|}{
    Costo: 7/giorno}                           \\\hline
\end{tabularx}

\begin{tabularx}{\linewidth}{|l|X|X|X|}
    \hline
    \multicolumn{4}{|>{\columncolor{tblhdrcolor}}c|}{
    \textbf{Operazione 24}}                     \\\hline
    \rowcolor{tblhdrcolor}
    \multicolumn{1}{|c|}{\textbf{Concetto}}
     & \multicolumn{1}{|c|}{\textbf{Costrutto}}
     & \multicolumn{1}{|c|}{\textbf{Accessi}}
     & \multicolumn{1}{|c|}{\textbf{Tipo}}
    \\ \hline
    Prenotazione Confermata
     & E
     & 1
     & L
    \\\hline
    Prenotazione Validata
     & E
     & 1
     & L
    \\\hline
    Prenotazione Scaduta
     & E
     & 1
     & L
    \\\hline
    Sala
     & E
     & 1
     & L
    \\\hline
    \multicolumn{4}{|>{\columncolor{tblhdrcolor}}l|}{
    Costo: 4/giorno}                            \\\hline
\end{tabularx}

\pagebreak
\subsection*{Ristrutturazione dello schema E-R}
%
% Descrivere (laddove necessario fornendo anche degli schemi) quali passi
% vengono adottati per ristrutturare lo schema E-R, ad esempio in
% termini di:
%   1. Analisi delle ridondanze
%   2. Eliminazione delle generalizzazioni
%   3. Scelta degli identificatori primari
% Si noti che in questa fase è possibile fare riferimento al costo delle
% operazioni precedentemente realizzato per guidare le scelte.
% Ad esempio, un leggero spreco di memoria legato alla non rimozione di
% ridondanze può essere facilmente giustificato da un guadagno in termini
% di prestazioni.
%

Poiché le operazioni di prenotazione e rimborso richiedono unicamente il codice
di prenotazione per per essere effettuate non è necessario richiedere ai clienti
di identificarsi per interagire con il sistema, ciò implica che i dati
contenuti nell'entità cliente e nella relazione ClientePrenotazione risultano
superflui. Rimuovendo le due entità in questione è possibile ottenere un
risparmio di memoria pari a %100 000 * (4 + 24 + 256) + 560 000 * 4 byte
31.4 MB circa ed un iterazione con il sistema semplificata per i clienti.

Poiché non esistono operazioni riguardanti le prenotazioni validate che
richiedono la maschera validante, i dati nella relazione Validazione risultano
superflui, rimuovendo la relazione è possibile ottenere un risparmio di memoria
pari a % 425 600 * 4 byte
1.6 MB circa.

Le entità Maschera e Proiezionista sono distinte dalla relazione
ProiezioneProiezionista, le operazioni che le riguardano fanno distinzione tra
le due ma il loro accorpamento non comporterebbe costi aggiuntivi.
Pertanto le entità figlie della generalizzazione verranno accorpate nel
genitore aggiungendo un attributo Ruolo all'entità Dipendente e collegando
l'entità ProiezioneProiezionista a Dipendente in quanto la presenza di valori
nulli nella relazione non comporterà particolari svantaggi.
L'aggiunta dell'attributo Ruolo comporta uno spreco di memoria irrisorio pari a
%200 * 1 byte 
200 byte che non è considerabile come un reale svantaggio.

%controllare il nuovo spreco di memoria

Le entità figlie dell'entità Prenotazione confermata non hanno attributi
specifici che le distinguono, le operazioni che le riguardano fanno distinzione 
tra i diversi tipi ma il loro accorpamento comporterebbe una riduzione dei
costi.
Pertanto le entità figlie della generalizzazione verranno accorpate nel
genitore aggiungendo un attributo Stato all'entità Prenotazione confermata.
l'Operazione 14 consisterà in un'unica scrittura riguardante
l'entità Prenotazione confermata ed il suo costo si ridurrà da 2000 a 1000
accessi al giorno,
l'Operazione 15 consisterà in un'unica scrittura riguardante
l'entità Prenotazione confermata ed il suo costo si ridurrà da 18000 a 9000
accessi al giorno,
per finire l'attributo Stato comporta uno spreco di memoria pari a
% 532 000 * 1 byte
520 kB circa che risulta accettabile se comparato al guadagno in termini di
prestazioni.

Le entità figlie dell'entità Prenotazione hanno ora attributi specifici 
distinti e le operazioni che le riguardano fanno distinzione tra i diversi
tipi ma il loro accorpamento comporterebbe una riduzione dei costi in accesso.
Pertanto le entità figlie della generalizzazione verranno accorpate nel
genitore aggiungendo gli attributi Stato, Codice transazione e Timestamp
all'entità Prenotazione.
In questo modo il costo dell'Operazione 12 verrà ridotto di un terzo,
passando da un costo di 15000 accessi a 10000 accessi al giorno mentre
lo spreco di memoria sarà pari a
% 56 000 * (1 + 4) byte + 560 000 * 4 byte
2.4 MB circa che risulta accettabile se comparato al guadagno in termini di
prestazioni.

L'attributo multivalore Cast viene coinvolto unicamente durante l'Operazione 11,
pertanto anziché introdurre un'entità Cast ed una relazione CastFilm
l'attributo multivalore verrà ridotto ad un semplice attributo Cast il cui
dominio sarà quello di una stringa.
In questo modo il costo dell'operazione 11 anziché incrementare a 15000
rimarrà di 10000 accessi al giorno.

È possibile ridurre il costo dell'Operazione 12 introducendo una relazione
PostoDisponibile tra le entità Proiezione e Posto con cardinalità (0,N)|(0,N)
riducendo il costo da 15000 a 5000 accessi al giorno.
La ridondanza comporta uno spreco di memoria pari a
% 1 400 * 200 * (4 + 1 + 4 + 4) byte
3.5 MB circa che risulta accettabile.
Tuttavia per poter mantenere coerenti i dati della relazione introdotta il costo
delle operazioni 13 e 14 raddoppia, portando il costo delle operazioni totali a
%10 000 + 1 000 - 10000
1000 accessi accessi aggiuntivi al giorno rispetto alla soluzione originale.
Pertanto non converrà introdurre la relazione PostoDisponibile

\includegraphics[width=\linewidth]{er_diagrams/restructuration.png}

\subsection*{Trasformazione di attributi e identificatori}
%
% Qualora siano presenti, in questa fase della progettazione, attributi
% ripetuti o identificatori esterni, descrivere quali trasformazioni
% vengono realizzate sul modello per facilitare la traduzione nello
% schema relazionale.
%

Poiché le entità Posto e Proiezione hanno gli attributi di Sala come
identificatori esterni, la tabella Prenotazione conterrà due volte gli attributi
Numero ed Id, rispettivamente delle tabelle Sala e Cinema.
Per evitare questa ridondanza la tabella Prenotazione conterrà un'unica coppia
di colonne corrispondente agli attributi Id, Numero in vincolo di integrità
referenziale sia con i corrispettivi della tabella Posto che con quelli di
Proiezione.

Sebbene le chiavi primarie di Posto e di Proiezione siano la composizione di
molti attributi, poiché le operazioni richieste sono particolarmente semplici
non sarà necessario sostituirli con un codice identificativo artificiale.

\pagebreak
\subsection*{Traduzione di entità e associazioni}
%
% Riportare in questa sezione la traduzione di entità ed associazioni
% nello schema relazionale.
% Fornire una rappresentazione grafica del modello relazionale completo.
%

Cinema(\underline{id}, indirizzo, orario)

Sale(\underline{cinema}, \underline{numero})

Posti(\underline{cinema},
\underline{sala},
\underline{fila},
\underline{numero})

Dipendente(\underline{matricola}, nome, cognome, ruolo)

Turni(\underline{dipendente}, \underline{giorno}, \underline{orario}, cinema)

Film(\underline{id}, nome, durata, casa\_cinematografica, cast)

Proiezioni(\underline{cinema},
\underline{sala},
\underline{data},
\underline{ora},
prezzo,
film,
proiezionista)

Prenotazioni(\underline{codice},
transazione,
stato,
cinema,
sala,
data,
ora,
fila,
numero,
timestamp)

\quad Con i vincoli:

Sale(cinema) $\subseteq$ Cinema(id)

Posti(cinema, sala) $\subseteq$ Sale(cinema, numero)

Turni(dipendente) $\subseteq$ Dipendente(matricola)

Turni(cinema) $\subseteq$ Cinema(id)

Proiezioni(cinema, sala) $\subseteq$ Sale(cinema, numero)

Proiezioni(film) $\subseteq$ Film(id)

Proiezioni(proiezionista) $\subseteq$ Dipendente(matricola)

Prenotazioni(cinema, sala, data, ora)
$\subseteq$
Proiezioni(cinema, sala, data, ora)

Prenotazioni(cinema, sala, fila, numero)
$\subseteq$
Posti(cinema, sala, fila, numero)

\subsection*{Normalizzazione del modello relazionale}
%
% Effettuare la normalizzazione del modello relazionale precedentemente
% descritto (in forma grafica) andando a mostrare le forme 1NF, 2NF, 3NF.
%

Notiamo che nessuna tabella presenta dipendenze funzionali non banali
$X \rightarrow Y$ tali che $X$ non sia una superchiave.

Notiamo inoltre che lo schema non contiene tabelle contenenti gruppi ripetuti
di attributi e ciascun attributo è definito su un dominio con valori atomici,
ricordiamo difatti che l'attributo Film(cast) è definito sul dominio delle
stringhe e non vi è nessun vincolo su di esso, pertanto i valori che può
assumere risultano atomici (la responsabilità di garantire la loro validità o
efficacia è lasciata all'amministratore).

La naturale assenza di dipendenze parziali e transitive che consegue dalla
prima osservazione ci permette quindi di concludere che lo schema è in BCNF.
