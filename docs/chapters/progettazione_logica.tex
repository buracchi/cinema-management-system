\section{Progettazione logica}

\subsection*{Volume dei dati}
%
% Questa sezione serve ad illustrare qual è il carico che la base di dati
% dovrà sopportare.
% A tal fine, è necessario prevedere un volume di dati attesi.
% Compilare la tabella sottostante, per ciascun concetto identificato nello
% schema E-R. I volumi devono essere stimati dallo studente in maniera
% ragionevole rispetto all’operatività presunta dell’applicativo.
%
\begin{tabularx}{\linewidth}{|X|l|X|}
    \hline
    \rowcolor{tblhdrcolor}
    \multicolumn{1}{|c|}{\textbf{Concetto nello schema}}
     & \multicolumn{1}{|c|}{\textbf{Tipo}
        \footnote{Indicare con E le entità, con R le relazioni}}
     & \multicolumn{1}{|c|}{\textbf{Volume atteso}}
    \\\hline
    \hfill
     & \hfill
     & \hfill
    \\ \hline
\end{tabularx}

\subsection*{Tavola delle operazioni}
%
% Rappresentare nella tabella sottostante tutte le operazioni sulla base
% di dati che devono essere supportate dall’applicazione, con la
% frequenza attesa.
% Le operazioni da supportare devono essere desunte dalle specifiche raccolte.
%
\begin{tabularx}{\linewidth}{|l|X|X|}
    \hline
    \rowcolor{tblhdrcolor}
    \multicolumn{1}{|c|}{\textbf{Cod.}}
     & \multicolumn{1}{|c|}{\textbf{Descrizione}}
     & \multicolumn{1}{|c|}{\textbf{Frequenza attesa}}
    \\\hline
    \hfill
     & \hfill
     & \hfill
    \\ \hline
\end{tabularx}

\subsection*{Costo delle operazioni}
%
% In riferimento a tutte le operazioni precedentemente indicate, calcolarne
% il costo supponendo, per questa fase del progetto, che il costo in
% scrittura di un dato sia doppio rispetto a quello in lettura.
%
\subsection*{Ristrutturazione dello schema E-R}
%
% Descrivere (laddove necessario fornendo anche degli schemi) quali passi
% vengono adottati per ristrutturare lo schema E-R, ad esempio in
% termini di:
%   1. Analisi delle ridondanze
%   2. Eliminazione delle generalizzazioni
%   3. Scelta degli identificatori primari
% Si noti che in questa fase è possibile fare riferimento al costo delle
% operazioni precedentemente realizzato per guidare le scelte.
% Ad esempio, un leggero spreco di memoria legato alla non rimozione di
% ridondanze può essere facilmente giustificato da un guadagno in termini
% di prestazioni.
%
\subsection*{Trasformazione di attributi e identificatori}
%
% Qualora siano presenti, in questa fase della progettazione, attributi
% ripetuti o identificatori esterni, descrivere quali trasformazioni
% vengono realizzate sul modello per facilitare la traduzione nello
% schema relazionale.
%
\subsection*{Traduzione di entità e associazioni}
%
% Riportare in questa sezione la traduzione di entità ed associazioni
% nello schema relazionale.
% Fornire una rappresentazione grafica del modello relazionale completo.
%
\subsection*{Normalizzazione del modello relazionale}
%
% Effettuare la normalizzazione del modello relazionale precedentemente
% descritto (in forma grafica) andando a mostrare le forme 1NF, 2NF, 3NF.
%
