\section{Progettazione logica}

\subsection*{Volume dei dati}

\begin{templateblock}
    Questa sezione serve ad illustrare qual è il carico che la base di dati
    dovrà sopportare.
    A tal fine, è necessario prevedere un volume di dati attesi.
    Compilare la tabella sottostante, per ciascun concetto identificato nello
    schema E-R. I volumi devono essere stimati dallo studente in maniera
    ragionevole rispetto all’operatività presunta dell’applicativo.
\end{templateblock}

\begin{tabularx}{\linewidth}{|Y|c|Y|}
    \hline
    \rowcolor{tgray}
    \textbf{Concetto nello schema} & \textbf{Tipo}\footnote{Indicare con E le entità, con R le relazioni} & \textbf{Volume atteso} \\\hline
                                   &                                                                      &                        \\ \hline
\end{tabularx}

\subsection*{Tavola delle operazioni}

\begin{templateblock}
    Rappresentare nella tabella sottostante tutte le operazioni sulla base
    di dati che devono essere supportate dall’applicazione, con la
    frequenza attesa.
    Le operazioni da supportare devono essere desunte dalle specifiche raccolte.
\end{templateblock}

\begin{tabularx}{\linewidth}{|c|Y|Y|}
    \hline
    \rowcolor{tgray}
    \textbf{Cod.} & \textbf{Descrizione} & \textbf{Frequenza attesa} \\\hline
                  &                      &                           \\ \hline
\end{tabularx}

\subsection*{Costo delle operazioni}

\begin{templateblock}
    In riferimento a tutte le operazioni precedentemente indicate che
    coinvolgono delle scritture (inserimenti e/o aggiornamenti), calcolarne
    il costo supponendo, per questa fase del progetto, che il costo in
    scrittura di un dato sia doppio rispetto a quello in lettura.
\end{templateblock}

\subsection*{Ristrutturazione dello schema E-R}

\begin{templateblock}
    Descrivere (laddove necessario fornendo anche degli schemi) quali passi
    vengono adottati per ristrutturare lo schema E-R, ad esempio in
    termini di:
    \begin{enumerate}
        \item Analisi delle ridondanze
        \item Eliminazione delle generalizzazioni
        \item Scelta degli identificatori primari
    \end{enumerate}
    Si noti che in questa fase è possibile fare riferimento al costo delle
    operazioni precedentemente realizzato per guidare le scelte.
    Ad esempio, un leggero spreco di memoria legato alla non rimozione di
    ridondanze può essere facilmente giustificato da un guadagno in termini
    di prestazioni.
\end{templateblock}

\subsection*{Trasformazione di attributi e identificatori}

\begin{templateblock}
    Qualora siano presenti, in questa fase della progettazione, attributi
    ripetuti o identificatori esterni, descrivere quali trasformazioni
    vengono realizzate sul modello per facilitare la traduzione nello
    schema relazionale.
\end{templateblock}

\subsection*{Traduzione di entità e associazioni}

\begin{templateblock}
    Riportare in questa sezione la traduzione di entità ed associazioni
    nello schema relazionale.

    Fornire una rappresentazione grafica del modello relazionale completo.
\end{templateblock}

\subsection*{Normalizzazione del modello relazionale}

\begin{templateblock}
    Effettuare la normalizzazione del modello relazionale precedentemente
    descritto (in forma grafica) andando a mostrare le forme 1NF, 2NF, 3NF.
\end{templateblock}
