\section{Progettazione logica}

\subsection*{Volume dei dati}
%
% Questa sezione serve ad illustrare qual è il carico che la base di dati
% dovrà sopportare.
% A tal fine, è necessario prevedere un volume di dati attesi.
% Compilare la tabella sottostante, per ciascun concetto identificato nello
% schema E-R. I volumi devono essere stimati dallo studente in maniera
% ragionevole rispetto all’operatività presunta dell’applicativo.
%
\begin{tabularx}{\linewidth}{|X|l|X|}
    \hline
    \rowcolor{tblhdrcolor}
    \multicolumn{1}{|c|}{\textbf{Concetto nello schema}}
     & \multicolumn{1}{|c|}{\textbf{Tipo}
        \footnote{Indicare con E le entità, con R le relazioni}}
     & \multicolumn{1}{|c|}{\textbf{Volume atteso}}
    \\\hline
    Cinema
     & E
     & 10
    \\\hline
    Sala
     & E
     & 50 (in media 5 sale per ogni cinema)
    \\ \hline
    Posto
     & E
     & 10 000 (in media 200 posti per ogni sala)
    \\ \hline
    Dipendente
     & E
     & 200
    \\ \hline
    Maschera
     & E
     & 50
    \\ \hline
    Proiezionista
     & E
     & 150
    \\ \hline
    Turno
     & E
     & 1 200 (in media un turno per ogni dipendente per sei giorni a settimana)
    \\ \hline
    Film
     & E
     & 600 000
    \\ \hline
    Proiezione
     & E
     & 1 400 (in media 4 proiezioni al giorno per ogni sala in una settimana
    prima del cambio del palinsesto)
    \\ \hline
    Prenotazione
     & E
     & 140 000 (in media la prenotazione del 50\% dei posti disponibili al
    giorno)
    \\ \hline
    Prenotazione annullata
     & E
     & 14 000 (in media il 10\% delle prenotazioni)
    \\ \hline
    Prenotazione confermata
     & E
     & 126 000 (in media il 90\% delle prenotazioni)
    \\ \hline
    Prenotazione validata
     & E
     & 113 400 (in media il 90\% delle prenotazioni confermate)
    \\ \hline
    Prenotazione scaduta
     & E
     & 12 600 (in media il 10\% delle prenotazioni confermate)
    \\ \hline
    CinemaSala
     & R
     & 50
    \\ \hline
    SalaPosto
     & R
     & 10 000
    \\ \hline
    ProiezioneSala
     & R
     & 1 400
    \\ \hline
    PostoPrenotazione
     & R
     & 140 000
    \\ \hline
    PrenotazioneProiezione
     & R
     & 140 000
    \\ \hline
    FilmProiezione
     & R
     & 14 000
    \\ \hline
    ProiezioneProiezionista
     & R
     & 14 000
    \\ \hline
    Validazione
     & R
     & 113 400
    \\ \hline
    DipendenteTurno
     & R
     & 1 200
    \\ \hline
    CinemaTurno
     & R
     & 1 200
    \\ \hline
\end{tabularx}

\subsection*{Tavola delle operazioni}
%
% Rappresentare nella tabella sottostante tutte le operazioni sulla base
% di dati che devono essere supportate dall’applicazione, con la
% frequenza attesa.
% Le operazioni da supportare devono essere desunte dalle specifiche raccolte.
%
\begin{tabularx}{\linewidth}{|l|X|X|}
    \hline
    \rowcolor{tblhdrcolor}
    \multicolumn{1}{|c|}{\textbf{Cod.}}
     & \multicolumn{1}{|c|}{\textbf{Descrizione}}
     & \multicolumn{1}{|c|}{\textbf{Frequenza attesa}}
    \\\hline
    \hfill
     & \hfill
     & \hfill
    \\ \hline
\end{tabularx}

\subsection*{Costo delle operazioni}
%
% In riferimento a tutte le operazioni precedentemente indicate, calcolarne
% il costo supponendo, per questa fase del progetto, che il costo in
% scrittura di un dato sia doppio rispetto a quello in lettura.
%
\subsection*{Ristrutturazione dello schema E-R}
%
% Descrivere (laddove necessario fornendo anche degli schemi) quali passi
% vengono adottati per ristrutturare lo schema E-R, ad esempio in
% termini di:
%   1. Analisi delle ridondanze
%   2. Eliminazione delle generalizzazioni
%   3. Scelta degli identificatori primari
% Si noti che in questa fase è possibile fare riferimento al costo delle
% operazioni precedentemente realizzato per guidare le scelte.
% Ad esempio, un leggero spreco di memoria legato alla non rimozione di
% ridondanze può essere facilmente giustificato da un guadagno in termini
% di prestazioni.
%
\subsection*{Trasformazione di attributi e identificatori}
%
% Qualora siano presenti, in questa fase della progettazione, attributi
% ripetuti o identificatori esterni, descrivere quali trasformazioni
% vengono realizzate sul modello per facilitare la traduzione nello
% schema relazionale.
%
\subsection*{Traduzione di entità e associazioni}
%
% Riportare in questa sezione la traduzione di entità ed associazioni
% nello schema relazionale.
% Fornire una rappresentazione grafica del modello relazionale completo.
%
\subsection*{Normalizzazione del modello relazionale}
%
% Effettuare la normalizzazione del modello relazionale precedentemente
% descritto (in forma grafica) andando a mostrare le forme 1NF, 2NF, 3NF.
%
