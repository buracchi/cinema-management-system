\section{Analisi dei Requisiti}
%
% Lo scopo di questa sezione è raffinare la specifica fornita, andando ad
% effettuare un'operazione preliminare di disambiguazione.
%
\subsection*{Identificazione dei termini ambigui e correzioni possibili}
%
% Compilare la seguente tabella, facendo riferimento alla specifica del
% minimondo di riferimento precedentemente indicata.
% Individuare i termini ambigui nella specifica
% (indicando la linea in cui essi compaiono), indicare il nuovo termine che
% si intende adottare nella specifica, ed indicare il motivo del cambiamento
% che si propone.
%
\begin{tabularx}{\linewidth}{|p{1.5cm}|p{3cm}|p{3cm}|X|}
    \hline
    \rowcolor{tblhdrcolor}
    \multicolumn{1}{|c|}{\textbf{Linea}}
     & \multicolumn{1}{|c|}{\textbf{Termine}}
     & \multicolumn{1}{|c|}{\textbf{Nuovo termine}}
     & \multicolumn{1}{|c|}{\textbf{Motivo correzione}}
    \\\hline
    1, 23
     & catena di cinema
     & catena cinematografica
     & Il termine "cinema" viene utilizzato per descrivere un’entità nel
    dominio di riferimento, il suo utilizzo nel descrivere un elemento
    differente crea pertanto un’ambiguità.
    \\ \hline
    4
     & amministrazione della catena
     & amministratori
     & Il termine "amministrazione" viene utilizzato come sinonimo di
    "amministratori" generando un'ambiguità.

    I termini "della catena" sono pleonastici.
    \\ \hline
    24, 25
     & amministratori della catena
     & amministratori
     & I termini "della catena" sono pleonastici.
    \\ \hline
    14, 19
     & utente
     & cliente
     & Il termine "utente" viene utilizzato come sinonimo di
    "cliente" generando un'ambiguità.
    \\ \hline
    32
     & spettatore
     & cliente
     & Il termine "spettatore" viene utilizzato come sinonimo di
    "cliente" generando un'ambiguità.
    \\ \hline
    14
     & visione
     & proiezione
     & Il termine "visione" viene utilizzato come sinonimo di
    "proiezione" generando un'ambiguità.
    \\ \hline
    26
     & spettacolo
     & proiezione
     & Il termine "spettacolo" viene utilizzato come sinonimo di
    "proiezione" generando un'ambiguità.
    \\ \hline
    11, 28, 30
     & biglietto
     & prenotazione
     & Il termine "biglietto" viene utilizzato come sinonimo di
    "prenotazione" generando un'ambiguità.
    \\ \hline
    10
     & biglietto utilizzato
     & prenotazione validata
     & Il termine "biglietto" viene utilizzato come sinonimo di
    "prenotazione" generando un'ambiguità.
    \\ \hline
    24
     & turni di lavoro
     & turni
     & I termini "di lavoro" sono pleonastici.
    \\ \hline
    10
     & cast degli attori protagonisti
     & cast
     & Utilizziamo per semplicità il termine "cast" per riferirsi al
    "cast degli attori protagonisti".
    \\ \hline
    16
     & perfezionare
     & completare
     & Il termine "perfezionare" è utilizzato come sinonimo di "completare"
    rendendo più difficile la comprensione del testo.
    \\ \hline
    27
     & verifica
     & validazione
     & Il termine "verifica" è utilizzato come sinonimo di "validazione"
    generando un'ambiguità.
    \\ \hline
\end{tabularx}

\pagebreak
\subsubsection*{Interlocuzione con il cliente}

\begin{longtable}{|p{8.15cm}|p{8.15cm}|}
    \hline
    \rowcolor{tblhdrcolor}
    \multicolumn{1}{|c|}{\textbf{Messaggio}}
     & \multicolumn{1}{|c|}{\textbf{Risposta}}
    \\\hline
    Non è chiaro il significato di questo periodo nella specifica:

    "L'amministrazione della catena gestisce i cinema."

    La frase in questione è circostanziale o con il termine "gestisce" si vuole
    far riferimento a delle operazioni specifiche sui cinema che non sono state
    definite e che si richiede di mettere a disposizione degli amministratori?
     & Definisce, ad esempio, chi è che può decidere il palinsesto dei cinema,
    ossia l'amministrazione della catena può cambiare ad esempio la
    programmazione

    Così come anche i turni di lavoro dei dipendenti
    \\\hline
    Quando gli amministratori modificano la programmazione delle proiezioni
    dovrebbe essere il sistema ad aggiornare coerentemente tutti i palinsesti
    o la modifica manuale dei palinsesti è un'operazione che si richiede
    di mettere a disposizione degli amministratori?
     & Non ho capito sinceramente in cosa potrebbe consistere l'automatismo...
    \\\hline
    Attraverso la procedura di aggiornamento della programmazione delle
    proiezioni gli amministratori possono aggiornare le proiezioni in programma.

    Durante la procedura di prenotazione il cliente ha la possibilità di
    verificare il palinsesto di un cinema per decidere quale proiezione
    prenotare.

    Queste due informazioni sono corrette?
     & Si
    \\\hline
    Il palinsesto di un cinema mostrato al cliente durante la procedura di
    prenotazione dev'essere generato dal sistema o dev'essere possibile per
    gli amministratori modificarlo manualmente?
     & Credo ci sia un equivoco: come può un sistema generare automaticamente
    qualcosa se non ha le informazioni per farlo? Il palinsesto è solo la
    lista dei film con gli orari. Se gli amministratori non inseriscono i
    film con gli orari, cosa può generare il sistema in automatico?
    \\\hline
    Nel caso in cui il sistema non contenga le informazioni necessarie per
    generare il palinsesto il sistema informerà il cliente che non sono
    presenti proiezioni in programma.
    Durante la procedura di aggiornamento della programmazione delle
    proiezioni gli amministratori dovranno inserire nel sistema i dati relativi
    alle proiezioni, che comprendono i film e gli orari di proiezione.
    Queste informazioni risiederanno dunque nel sistema al termine della prima
    procedura di aggiornamento della programmazione delle proiezioni.
     & Continuo a non capire qual è il punto di tutta questa discussione...
    U+1F601

    Se non ci sono film, i film prenotabili sono zero, non vedo cosa cosa ci
    sia di diverso nell'"informare il cliente" di questo dal mostrare i film
    che può prenotare U+1F914

    In ultimo sì: sono gli amministratori che possono decidere la
    programmazione dei film, come dicevamo già prima!
    \\\hline
    Nella specifica si fa riferimento al fatto che i cinema hanno delle
    fasce orarie d'apertura. Può fornirmi più dettagli su quest'ultime?
    Devono rispettare dei vincoli?
     & Non è necessario che il sistema imponga dei vincoli, anzi è preferibile
    che non ne imponga affatto, poiché non sappiamo oggi quali orari potremo
    avere nei nostri cinema in futuro.
    Quindi, la cosa migliore è lasciare la libertà agli amministratori di
    definire gli orari di apertura come meglio credono.
    \\\hline
    Nella specifica si fa riferimento a delle prenotazioni "confermate".

    Cosa si intende per "confermate"?
     & Si fa riferimento a questo
    "Dal momento dell’inizio della procedura di prenotazione, un cliente ha
    a disposizione 10 minuti per perfezionare la prenotazione"

    e a questo "Fino a 30 minuti dall'inizio della proiezione, il cliente ha
    la possibilità di annullare la sua prenotazione fornendo al sistema il
    codice di prenotazione."
    \\\hline
    Nella specifica si fa poi riferimento a delle
    prenotazioni confermate " non  [...] utilizzate per accedere al cinema".

    Cosa si intende per "non utilizzate"? Che non sono ancora state
    validate o che sono scadute?
     & Sì, sono i biglietti venduti a persone che poi non sono andate al cinema
    \\\hline
    Nella specifica si fa riferimento al fatto che
    "I turni sono gestiti su base settimanale".

    Cosa si intende con questo?
     & Che ogni settimana si fanno i turni
    \\\hline
    Lo stesso dipendente può lavorare in cinema diversi in turni diversi o
    lavora unicamente in un cinema?
     & Questo non è detto, può essere, non vorrei che ci fossero vincoli dal
    sistema: ce la gestiamo noi quando facciamo i turni
    \\\hline
\end{longtable}

\pagebreak
\subsubsection*{Specifica disambiguata}

\begin{longtable}{|p{\linewidth}|}
    \hline
    % Riportare in questo riquadro la specifica di progetto corretta,
    % applicando le disambiguazioni proposte.
    \begin{itemize}
        \item Ogni cinema ha un numero arbitrario di sale.
        \item Ogni cinema ha una fascia oraria di apertura.
        \item Ogni sala è identificata da un numero di sala.
        \item Ogni sala ha un numero arbitrario di posti.
        \item Ogni posto è identificato da una lettera per la fila ed un numero
              di posto.
        \item In ogni sala si possono proiettare più film nello stesso giorno.
        \item Ogni film può essere proiettato più volte nello stesso giorno ed
              in sale differenti.
        \item Ogni film ha una durata, un nome, un cast e una
              casa cinematografica.
        \item Il prezzo di una prenotazione dipende dal film proiettato,
              dalla sala e dall'orario in cui avviene la proiezione.
        \item Gli amministratori gestiscono la programmazione delle proiezioni.
        \item I clienti possono prenotare un posto per una proiezione
              completando la procedura di prenotazione.
        \item I clienti hanno a disposizione 10 minuti per completare la
              procedura di prenotazione.
        \item La procedura di prenotazione è formata dai seguenti passaggi:
              \begin{enumerate}
                  \item il cliente seleziona un cinema
                  \item il sistema permette al cliente di verificare
                        il palinsesto del cinema selezionato
                  \item il cliente seleziona una proiezione
                  \item il sistema permette al cliente di verificare
                        i posti disponibili
                  \item il cliente seleziona un posto disponibile
                  \item il cliente inserire i dati relativi alla carta
                        di credito (numero, intestatario, data di scadenza,
                        codice CVV)
                  \item il sistema richiede al servizio di pagamento di
                        generare la relativa transazione
                  \item il servizio di pagamento restituisce al sistema
                        il codice della transazione generata
                  \item il sistema genera la prenotazione e la registra
                        come prenotazione confermata
                  \item il sistema restituisce al cliente un codice di
                        prenotazione
              \end{enumerate}
        \item I clienti possono annullare la propria prenotazione completando
              la procedura di annullamento di una prenotazione
        \item I clienti possono annullare la propria prenotazione
              entro 30 minuti dall'inizio della proiezione.
        \item La procedura di annullamento di una prenotazione è formata dai
              seguenti passaggi:
              \begin{enumerate}
                  \item il cliente fornisce al sistema un codice di
                        prenotazione
                  \item il sistema richiede al servizio di pagamento di
                        annullare la relativa transazione
                  \item il sistema registra la prenotazione come prenotazione
                        annullata
              \end{enumerate}
        \item I dipendenti sono divisi in maschere e proiezionisti.
        \item Ogni dipendente può effettuare un turno in un qualunque cinema.
        \item Gli amministratori gestiscono la programmazione dei turni.
        \item La durata di una giornata lavorativa di un dipendente deve essere
              di massimo otto ore.
        \item I turni sono gestiti su base settimanale.
        \item Gli amministratori possono generare un report che permette di
              sapere se:
              \begin{itemize}
                  \item qualche proiezione è sprovvista di proiezionista
                  \item qualche cinema, in qualche fascia oraria di apertura,
                        è sprovvisto di almeno due maschere
              \end{itemize}
    \end{itemize}
    \\\hline
    \begin{itemize}
        \item Le maschere validano le prenotazioni mediante il codice
              di prenotazione presente sui biglietti.
        \item Una prenotazione validata viene associata al fatto che quel
              determinato posto, per quella determinata proiezione,
              è stato occupato dal cliente.
        \item Non è possibile utilizzare più volte lo stesso codice di
              prenotazione per accedere al cinema.
        \item Gli amministratori possono generare dei report mensili che
              mostrano per ogni cinema e per ogni sala:
              \begin{itemize}
                  \item quante sono le prenotazioni confermate
                  \item quante sono le prenotazioni annullate
                  \item quante sono le prenotazioni scadute
              \end{itemize}
    \end{itemize}
    \\\hline
\end{longtable}

\pagebreak

\subsection*{Glossario dei Termini}
%
% Realizzare un dizionario dei termini, compilando la tabella qui sotto,
% a partire dalle specifiche precedentemente disambiguate
%
\begin{longtable}{|p{3.86cm}|p{3.86cm}|p{3.86cm}|p{3.86cm}|}
    \hline
    \rowcolor{tblhdrcolor}
    \multicolumn{1}{|c|}{\textbf{Termine}}
     & \multicolumn{1}{|c|}{\textbf{Descrizione}}
     & \multicolumn{1}{|c|}{\textbf{Sinonimi}}
     & \multicolumn{1}{|c|}{\textbf{Collegamenti}}
    \\\hline
    Cinema
     & Un cinema facente parte della catena cinematografica.
     &
     & Sala, Fascia oraria di apertura
    \\\hline
    Fascia oraria di apertura
     & Periodo temporale durante il quale il relativo cinema è aperto. [...]
     &
     &
    \\ \hline
    Sala
     & Una sala di un cinema in cui vengono effettuate proiezioni.
     &
     & Numero di sala, Posto
    \\\hline
    Numero di sala
     & Numero intero identificante una sala in un cinema.
     &
     &
    \\\hline
    Posto
     & Sedia, poltrona e sim. in una sala prenotabile da un cliente
    per una proiezione.
     &
     & Numero di posto
    \\\hline
    Numero di posto
     & Numero intero identificante la colonna del posto.
     &
     &
    \\\hline
    Film
     & Produzione cinematografica proiettabile in una sala di un cinema.
     &
     & Cast, Casa cinematografica
    \\\hline
    Cast
     & Insieme degli attori principali che lavora nella produzione di un film.
     & Cast degli attori protagonisti
     &
    \\\hline
    Casa cinematografica
     & Impresa che produce film.
     &
     &
    \\ \hline
    Proiezione
     & Riproduzione di un film in una sala in una certa data e in un
    certo orario.
     & Visione, Spettacolo
     & Sala, Film
    \\ \hline
    Programmazione delle proiezioni
     & La pianificazione delle proiezioni future.
     & Palinsesto
     &
    \\ \hline
    Palinsesto
     & Prospetto della programmazione delle proiezioni di un cinema.
     &
     & Programmazione
    \\ \hline
    Cliente
     & Utente che può effettuare la procedura di prenotazione.
     & Utente, Spettatore
     &
    \\ \hline
    Biglietto
     & Titolo di accesso ad un cinema, contiene i dati di una prenotazione
    ed è valido per la relativa proiezione.
     &
     & Prenotazione, Codice di prenotazione
    \\ \hline
    Prenotazione
     & Diritto a sedere su un certo posto per una certa proiezione acquisibile
    completando la procedura di prenotazione, il suo prezzo varia in base alla
    proiezione.
     & Biglietto
     & Cliente, Proiezione, Posto, Codice di Prenotazione, Carta di credito
    \\ \hline
    Codice di prenotazione
     & Codice univoco identificante una prenotazione.
    Permette di validare il relativo biglietto o di annullare la
    relativa prenotazione.
     &
     &
    \\ \hline
    Procedura di prenotazione
     & Processo tramite cui un cliente può effettuare una prenotazione.
     &
     & Cliente, Prenotazione, Posto disponibile, Codice di prenotazione,
    Carta di credito
    \\ \hline
    Posto disponibile
     & In riferimento ad una certa proiezione, posto non prenotato.
     &
     & Posto, Proiezione, Prenotazione
    \\ \hline
    Carta di credito
     & Strumento che consente all’intestatario di ottenere l'addebitamento
    del prezzo di beni e servizi con enti convenzionati con l’emittente.
     &
     &
    \\ \hline
    Biglietto utilizzato
     & Un biglietto associato ad una prenotazione validata.
     &
     & Biglietto, Codice di prenotazione
    \\ \hline
    Prenotazione confermata
     & Una prenotazione non annullata.
     &
     & Prenotazione
    \\ \hline
    Prenotazione annullata
     & Una prenotazione che è stata annullata dal cliente.
     &
     & Prenotazione
    \\ \hline
    Prenotazione validata
     & Una prenotazione confermata validata da una maschera.
     &
     & Prenotazione, Biglietto
    \\ \hline
    Prenotazione scaduta
     & Una prenotazione confermata relativa ad proiezione effettuata
     non validata da una maschera.
     &
     & Prenotazione, Biglietto
    \\ \hline
    Amministratore
     & Un amministratore della catena cinematografica.
     & Amministrazione della catena, Amministratori della catena
     &
    \\ \hline
    Dipendente
     & Un dipendente della catena cinematografica.
     &
     & Turno
    \\ \hline
    Maschera
     & Un tipo di dipendente.
     &
     &
    \\ \hline
    Proiezionista
     & Un tipo di dipendente.
     &
     &
    \\ \hline
    Turno
     & Un'intervallo di tempo di massimo otto ore all'interno di una giornata
    che definisce il periodo lavorativo di un dipendete.
     & Turno di lavoro
     &
    \\ \hline
    Programmazione dei turni
     & La pianificazione dei turni dei dipendenti svolta settimanalmente.
     & Palinsesto
     &
    \\ \hline
\end{longtable}

\subsection*{Raggruppamento dei requisiti in insiemi omogenei}
%
% Per ciascun elemento "più importante" della specifica
% (riportata anche nel glossario precedente), estrapolare dalla specifica
% disambiguata le frasi ad esso associate.
% Compilare una tabella separata per ciascun elemento individuato.
%
\begin{tabularx}{\linewidth}{|X|}
    \hline
    \rowcolor{tblhdrcolor}
    \textbf{Frasi relative ai cinema} \\\hline
    Ogni cinema ha un numero arbitrario di sale.

    Ogni cinema ha una fascia oraria di apertura.

    Ogni dipendente può effettuare un turno in un qualunque cinema.

    Gli amministratori possono generare un report che permette di
    sapere se:
    \begin{itemize}
        \item qualche proiezione è sprovvista di proiezionista
        \item qualche cinema, in qualche fascia oraria di apertura,
              è sprovvisto di almeno due maschere
    \end{itemize}
    \\ \hline
\end{tabularx}

\begin{tabularx}{\linewidth}{|X|}
    \hline
    \rowcolor{tblhdrcolor}
    \textbf{Frasi relative alle sale} \\\hline
    Ogni cinema ha un numero arbitrario di sale.

    Ogni sala è identificata da un numero di sala.

    Ogni sala ha un numero arbitrario di posti.

    In ogni sala si possono proiettare più film nello stesso giorno.

    Ogni film può essere proiettato più volte nello stesso giorno ed
    in sale differenti.

    Il prezzo di una prenotazione dipende dal film proiettato,
    dalla sala e dall'orario in cui avviene la proiezione.
    \\ \hline
\end{tabularx}

\begin{tabularx}{\linewidth}{|X|}
    \hline
    \rowcolor{tblhdrcolor}
    \textbf{Frasi relative ai posti} \\\hline
    Ogni posto è identificato da una lettera per la fila ed un numero
    di posto.

    I clienti possono prenotare un posto per una proiezione completando la
    procedura di prenotazione.

    Una prenotazione validata viene associata al fatto che quel determinato
    posto, per quella determinata proiezione, è stato occupato dal cliente.
    \\ \hline
\end{tabularx}

\begin{tabularx}{\linewidth}{|X|}
    \hline
    \rowcolor{tblhdrcolor}
    \textbf{Frasi relative ai film} \\\hline
    In ogni sala si possono proiettare più film nello stesso giorno.

    Ogni film può essere proiettato più volte nello stesso giorno ed
    in sale differenti.

    Ogni film ha una durata, un nome, un cast e una
    casa cinematografica.

    Il prezzo di una prenotazione dipende dal film proiettato,
    dalla sala e dall'orario in cui avviene la proiezione.
    \\ \hline
\end{tabularx}

\begin{tabularx}{\linewidth}{|X|}
    \hline
    \rowcolor{tblhdrcolor}
    \textbf{Frasi relative alle proiezioni} \\\hline
    In ogni sala si possono proiettare più film nello stesso giorno.

    Ogni film può essere proiettato più volte nello stesso giorno ed in sale
    differenti.

    Il prezzo di una prenotazione dipende dal film proiettato,
    dalla sala e dall'orario in cui avviene la proiezione.

    I clienti possono prenotare un posto per una proiezione
    completando la procedura di prenotazione.

    Una prenotazione validata viene associata al fatto che quel
    determinato posto, per quella determinata proiezione,
    è stato occupato dal cliente.
    \\ \hline
\end{tabularx}

\begin{tabularx}{\linewidth}{|X|}
    \hline
    \rowcolor{tblhdrcolor}
    \textbf{Frasi relative alla programmazione delle proiezioni} \\\hline

    In ogni sala si possono proiettare più film nello stesso giorno.

    Ogni film può essere proiettato più volte nello stesso giorno ed in sale
    differenti.

    Il prezzo di una prenotazione dipende dal film proiettato,
    dalla sala e dall'orario in cui avviene la proiezione.

    Gli amministratori gestiscono la programmazione delle proiezioni.
    \\ \hline
\end{tabularx}

\begin{tabularx}{\linewidth}{|X|}
    \hline
    \rowcolor{tblhdrcolor}
    \textbf{Frasi relative alle prenotazioni} \\\hline
    Il prezzo di una prenotazione dipende dal film proiettato,
    dalla sala e dall'orario in cui avviene la proiezione.

    I clienti possono prenotare un posto per una proiezione
    completando la procedura di prenotazione.

    Le maschere validano le prenotazioni mediante il codice
    di prenotazione presente sui biglietti.

    Una prenotazione validata viene associata al fatto che quel
    determinato posto, per quella determinata proiezione,
    è stato occupato dal cliente.

    Non è possibile utilizzare più volte lo stesso codice di
    prenotazione per accedere al cinema.

    Gli amministratori possono generare dei report mensili che
    mostrano per ogni cinema e per ogni sala:
    \begin{itemize}
        \item quante sono le prenotazioni confermate
        \item quante sono le prenotazioni annullate
        \item quante sono le prenotazioni scadute
    \end{itemize}
    \\ \hline
\end{tabularx}

\begin{tabularx}{\linewidth}{|X|}
    \hline
    \rowcolor{tblhdrcolor}
    \textbf{Frasi relative a procedura di prenotazione} \\\hline
    I clienti possono prenotare un posto per una proiezione
    completando la procedura di prenotazione.

    I clienti hanno a disposizione 10 minuti per completare la
    procedura di prenotazione.

    La procedura di prenotazione è formata dai seguenti passaggi:
    \begin{enumerate}
        \item il cliente seleziona un cinema
        \item il sistema permette al cliente di verificare
              il palinsesto del cinema selezionato
        \item il cliente seleziona una proiezione
        \item il cliente seleziona un posto disponibile
        \item il cliente inserire i dati relativi alla carta
              di credito (numero, intestatario, data di scadenza,
              codice CVV)
        \item il sistema restituisce al cliente un codice di
              prenotazione
    \end{enumerate}

    I clienti possono annullare la propria prenotazione usando il
    codice di prenotazione entro 30 minuti dall'inizio della
    proiezione.
    \\ \hline
\end{tabularx}

\begin{tabularx}{\linewidth}{|X|}
    \hline
    \rowcolor{tblhdrcolor}
    \textbf{Frasi relative agli amministratori} \\\hline
    Gli amministratori gestiscono la programmazione delle proiezioni.

    Gli amministratori gestiscono la programmazione dei turni.

    Gli amministratori possono generare un report che permette di
    sapere se:
    \begin{itemize}
        \item qualche proiezione è sprovvista di proiezionista
        \item qualche cinema, in qualche fascia oraria di apertura,
              è sprovvisto di almeno due maschere
    \end{itemize}

    Gli amministratori possono generare dei report mensili che
    mostrano per ogni cinema e per ogni sala:
    \begin{itemize}
        \item quante sono le prenotazioni confermate
        \item quante sono le prenotazioni annullate
        \item quante sono le prenotazioni scadute
    \end{itemize}
    \\ \hline
\end{tabularx}

\begin{tabularx}{\linewidth}{|X|}
    \hline
    \rowcolor{tblhdrcolor}
    \textbf{Frasi relative ai dipendenti} \\\hline
    I dipendenti sono divisi in maschere e proiezionisti.

    Ogni dipendente può effettuare un turno in un qualunque cinema.
    \\ \hline
\end{tabularx}

\begin{tabularx}{\linewidth}{|X|}
    \hline
    \rowcolor{tblhdrcolor}
    \textbf{Frasi relative alle maschere} \\\hline
    I dipendenti sono divisi in maschere e proiezionisti.

    Le maschere validano le prenotazioni mediante il codice
    di prenotazione presente sui biglietti.

    Gli amministratori possono generare un report che permette di
    sapere se:
    \begin{itemize}
        \item qualche proiezione è sprovvista di proiezionista
        \item qualche cinema, in qualche fascia oraria di apertura,
              è sprovvisto di almeno due maschere
    \end{itemize}
    \\ \hline
\end{tabularx}

\begin{tabularx}{\linewidth}{|X|}
    \hline
    \rowcolor{tblhdrcolor}
    \textbf{Frasi relative ai proiezionisti} \\\hline
    I dipendenti sono divisi in maschere e proiezionisti.

    Gli amministratori possono generare un report che permette di
    sapere se:
    \begin{itemize}
        \item qualche proiezione è sprovvista di proiezionista
        \item qualche cinema, in qualche fascia oraria di apertura,
              è sprovvisto di almeno due maschere
    \end{itemize}
    \\ \hline
\end{tabularx}

\begin{tabularx}{\linewidth}{|X|}
    \hline
    \rowcolor{tblhdrcolor}
    \textbf{Frasi relative ai turni} \\\hline
    Ogni dipendente può effettuare un turno in un qualunque cinema.

    I turni devono essere di massimo otto ore.

    I turni sono gestiti su base settimanale.
    \\ \hline
\end{tabularx}

\begin{tabularx}{\linewidth}{|X|}
    \hline
    \rowcolor{tblhdrcolor}
    \textbf{Frasi relative alla programmazione dei turni} \\\hline
    Gli amministratori gestiscono la programmazione dei turni.

    I turni sono gestiti su base settimanale.
    \\ \hline
\end{tabularx}
