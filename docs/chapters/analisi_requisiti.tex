\section{Analisi dei Requisiti}
%
% Lo scopo di questa sezione è raffinare la specifica fornita, andando ad
% effettuare un'operazione preliminare di disambiguazione.
%

\subsection*{Identificazione dei termini ambigui e correzioni possibili}
%
% Compilare la seguente tabella, facendo riferimento alla specifica del
% minimondo di riferimento precedentemente indicata.
% Individuare i termini ambigui nella specifica
% (indicando la linea in cui essi compaiono), indicare il nuovo termine che
% si intende adottare nella specifica, ed indicare il motivo del cambiamento
% che si propone.
%
\begin{tabularx}{\linewidth}{|p{1.5cm}|p{3cm}|p{3cm}|X|}
      \hline
      \rowcolor{tblhdrcolor}
      \multicolumn{1}{|c|}{\textbf{Linea}}
       & \multicolumn{1}{|c|}{\textbf{Termine}}
       & \multicolumn{1}{|c|}{\textbf{Nuovo termine}}
       & \multicolumn{1}{|c|}{\textbf{Motivo correzione}}
      \\\hline
      1, 23
       & catena di cinema
       & catena cinematografica
       & Il termine "cinema" viene utilizzato per descrivere un’entità nel
      dominio di riferimento, il suo utilizzo nel descrivere un elemento
      differente crea pertanto un’ambiguità.
      \\ \hline
      4
       & amministrazione della catena
       & amministratori
       & Il termine "amministrazione" viene utilizzato come sinonimo di
      "amministratori" generando un'ambiguità.

      I termini "della catena" sono pleonastici.
      \\ \hline
      6
       & ciascun
       & ogni posto
       & Il termine "ciascun" viene utilizzato per indicare in maniera implicita
      "ogni posto" complicando la comprensione del testo.
      \\ \hline
      24, 25
       & amministratori della catena
       & amministratori
       & I termini "della catena" sono pleonastici.
      \\ \hline
      14, 19
       & utente
       & cliente
       & Il termine "utente" viene utilizzato come sinonimo di
      "cliente" generando un'ambiguità.
      \\ \hline
      32
       & spettatore
       & cliente
       & Il termine "spettatore" viene utilizzato come sinonimo di
      "cliente" generando un'ambiguità.
      \\ \hline
      14
       & visione
       & proiezione
       & Il termine "visione" viene utilizzato come sinonimo di
      "proiezione" generando un'ambiguità.
      \\ \hline
      26
       & spettacolo
       & proiezione
       & Il termine "spettacolo" viene utilizzato come sinonimo di
      "proiezione" generando un'ambiguità.
      \\ \hline
      11, 28, 30
       & biglietto
       & prenotazione
       & Il termine "biglietto" viene utilizzato come sinonimo di
      "prenotazione" generando un'ambiguità.
      \\ \hline
      31
       & biglietto utilizzato
       & prenotazione validata
       & Il termine "biglietto" viene utilizzato come sinonimo di
      "prenotazione" generando un'ambiguità.
      \\ \hline
      24
       & turni di lavoro
       & turni
       & I termini "di lavoro" sono pleonastici.
      \\ \hline
      10
       & cast degli attori protagonisti
       & cast
       & Utilizziamo per semplicità il termine "cast" per riferirsi al
      "cast degli attori protagonisti".
      \\ \hline
      16
       & perfezionare
       & completare
       & Il termine "perfezionare" è utilizzato come sinonimo di "completare"
      complicando la comprensione del testo.
      \\ \hline
      27
       & verifica
       & validazione
       & Il termine "verifica" è utilizzato come sinonimo di "validazione"
      generando un'ambiguità.
      \\ \hline
      11
       & costo
       & prezzo
       & Il termine "costo" viene utilizzato per riferirsi al prezzo.
      \\ \hline
\end{tabularx}

\pagebreak
\subsubsection*{Specifica disambiguata}

\resetlinenumber[1]
\begin{tabularx}{\linewidth}{r|>{\internallinenumbers}X|}
      \cline{2-2} &
      % Riportare in questo riquadro la specifica di progetto corretta,
      % applicando le disambiguazioni proposte.
      Si vuole realizzare il sistema informativo di una catena cinematografica,
      che si occupi anche della gestione delle prenotazioni.
      \newline
      \newline
      Gli amministratori gestiscono i cinema.
      Ciascun cinema ha un numero arbitrario di sale, identificato da un
      numero di sala.
      In ogni sala c’è un numero arbitrario di posti, \linebreak
      ogni posto è individuato da una lettera per la fila ed un numero
      di posto.
      \newline
      \newline
      In ogni sala vengono proiettati più film quotidianamente.
      Ciascun cinema può proiettare lo stesso film più volte in una giornata,
      in sale differenti. Ogni film ha una durata, un nome, unl cast e una
      casa cinematografica.
      Lo stesso film, proiettato in orari differenti e in sale differenti,
      può avere un prezzo per la prenotazione differente, proprio in relazione
      alla sala e all’orario in cui esso viene proiettato.
      \newline
      \newline
      Il sistema di prenotazione è tale per cui i clienti possono prenotarsi
      per una proiezione, scegliendo un posto disponibile. Dal momento
      dell’inizio della procedura di prenotazione, un cliente ha a disposizione
      10 minuti per completare la prenotazione.
      Dopo aver scelto il  posto, al cliente è data la possibilità di inserire
      i dati relativi alla propria carta di credito
      (numero, intestatario, data di scadenza, codice CVV).
      Una volta inseriti questi dati, il sistema restituisce al cliente
      un codice di prenotazione.
      Fino a 30 minuti dall'inizio della proiezione, il cliente ha la
      possibilità di annullare la sua prenotazione fornendo al sistema il codice
      di prenotazione.
      \newline
      \newline
      La catena cinematografica gestisce anche i propri dipendenti, divisi in
      maschere e proiezionisti.
      Gli amministratori definiscono i turni, di otto ore massimo.
      I turni sono gestiti su base settimanale.
      Un report permette di sapere agli amministratori se qualche proiezione
      è sprovvista di proiezionista o se qualche cinema, in qualche
      fascia oraria di apertura, è sprovvisto di almeno due maschere per la
      verifica delle prenotazioni all’ingresso.
      \newline
      \newline
      La validazione delle prenotazioni avviene da parte delle maschere
      mediante l’utilizzo del codice di prenotazione.
      Una prenotazione validata viene associato nel sistema al fatto che quel
      determinato posto, per quella determinata proiezione, è stato occupato
      dal cliente.
      Non è possibile utilizzare più volte lo stesso codice di prenotazione
      per accedere al cinema.
      \newline
      \newline
      A fini statistici, gli amministratori possono generare dei report mensili
      che mostrano per ciascun cinema e ciascuna sala quante prenotazioni sono
      state confermate, quante sono state annullate, e quante prenotazioni
      confermate non sono state utilizzate per accedere al cinema.
      \\\cline{2-2}
\end{tabularx}

\pagebreak

\subsection*{Interazione col cliente}

\begin{longtable}{|p{1cm}|p{7.66cm}|p{7.66cm}|}
      \hline
      \rowcolor{tblhdrcolor}
      \multicolumn{1}{|c|}{\textbf{Id}}                                       &
      \multicolumn{1}{|c|}{\textbf{Messaggio}}                                &
      \multicolumn{1}{|c|}{\textbf{Risposta}}
      \\\hline
      I1                                                                      &
      Non è chiaro il significato di questo periodo nella specifica:

      "L'amministrazione della catena gestisce i cinema."

      La frase in questione è circostanziale o con il termine "gestisce"
      si vuole far riferimento a delle operazioni specifiche sui cinema che
      non sono state definite e che si richiede di mettere a disposizione
      degli amministratori?                                                   &
      Definisce, ad esempio, chi è che può decidere il palinsesto dei cinema,
      ossia l'amministrazione della catena può cambiare ad esempio la
      programmazione

      Così come anche i turni di lavoro dei dipendenti
      \\\hline
      I2                                                                      &
      Quando gli amministratori modificano la programmazione delle proiezioni
      dovrebbe essere il sistema ad aggiornare coerentemente tutti i palinsesti
      o la modifica manuale dei palinsesti è un'operazione che si richiede
      di mettere a disposizione degli amministratori?                         &
      Non ho capito sinceramente in cosa potrebbe consistere l'automatismo...
      \\\hline
      I3                                                                      &
      Attraverso la procedura di aggiornamento della programmazione delle
      proiezioni gli amministratori possono aggiornare le proiezioni in programma.

      Durante la procedura di prenotazione il cliente ha la possibilità di
      verificare il palinsesto di un cinema per decidere quale proiezione
      prenotare.

      Queste due informazioni sono corrette?                                  &
      Si
      \\\hline
      I4                                                                      &
      Il palinsesto di un cinema mostrato al cliente durante la procedura di
      prenotazione dev'essere generato dal sistema o dev'essere possibile per
      gli amministratori modificarlo manualmente?                             &
      Credo ci sia un equivoco: come può un sistema generare automaticamente
      qualcosa se non ha le informazioni per farlo? Il palinsesto è solo la
      lista dei film con gli orari. Se gli amministratori non inseriscono i
      film con gli orari, cosa può generare il sistema in automatico?
      \\\hline
      I5                                                                      &
      Nel caso in cui il sistema non contenga le informazioni necessarie per
      generare il palinsesto il sistema informerà il cliente che non sono
      presenti proiezioni in programma.
      Durante la procedura di aggiornamento della programmazione delle
      proiezioni gli amministratori dovranno inserire nel sistema i dati
      relativi alle proiezioni, che comprendono i film e gli orari di
      proiezione.
      Queste informazioni risiederanno dunque nel sistema al termine della
      prima procedura di aggiornamento della programmazione delle proiezioni. &
      Continuo a non capire qual è il punto di tutta questa discussione...
      U+1F601

      Se non ci sono film, i film prenotabili sono zero, non vedo cosa cosa ci
      sia di diverso nell'"informare il cliente" di questo dal mostrare i film
      che può prenotare U+1F914

      In ultimo sì: sono gli amministratori che possono decidere la
      programmazione dei film, come dicevamo già prima!
      \\\hline
      I6                                                                      &
      Nella specifica si fa riferimento al fatto che i cinema hanno delle
      fasce orarie d'apertura. Può fornirmi più dettagli su quest'ultime?
      Devono rispettare dei vincoli?                                          &
      Non è necessario che il sistema imponga dei vincoli, anzi è preferibile
      che non ne imponga affatto, poiché non sappiamo oggi quali orari potremo
      avere nei nostri cinema in futuro.
      Quindi, la cosa migliore è lasciare la libertà agli amministratori di
      definire gli orari di apertura come meglio credono.
      \\\hline
      I7                                                                      &
      Nella specifica si fa riferimento a delle prenotazioni "confermate".

      Cosa si intende per "confermate"?                                       &
      Si fa riferimento a questo
      "Dal momento dell’inizio della procedura di prenotazione, un cliente ha
      a disposizione 10 minuti per perfezionare la prenotazione"

      e a questo "Fino a 30 minuti dall'inizio della proiezione, il cliente ha
      la possibilità di annullare la sua prenotazione fornendo al sistema il
      codice di prenotazione."
      \\\hline
      I8                                                                      &
      Nella specifica si fa poi riferimento a delle
      prenotazioni confermate " non  [...] utilizzate per accedere al cinema".

      Cosa si intende per "non utilizzate"? Che non sono ancora state
      validate o che sono scadute?                                            &
      Sì, sono i biglietti venduti a persone che poi non sono andate al cinema
      \\\hline
      I9                                                                      &
      Nella specifica si fa riferimento al fatto che
      "I turni sono gestiti su base settimanale".

      Cosa si intende con questo?                                             &
      Che ogni settimana si fanno i turni
      \\\hline
      I10                                                                     &
      Lo stesso dipendente può lavorare in cinema diversi in turni diversi o
      lavora unicamente in un cinema?                                         &
      Questo non è detto, può essere, non vorrei che ci fossero vincoli dal
      sistema: ce la gestiamo noi quando facciamo i turni
      \\\hline
      I10                                                                     &
      Lo stesso dipendente può lavorare in cinema diversi in turni diversi o
      lavora unicamente in un cinema?                                         &
      Questo non è detto, può essere, non vorrei che ci fossero vincoli dal
      sistema: ce la gestiamo noi quando facciamo i turni
      \\\hline
      I11                                                                     &
      Due biglietti per lo stesso film proiettato nella stessa sala alla
      stessa ora ma in giornate differenti possono avere costi differenti?    &
      Sì, ad esempio il mercoledì i biglietti spesso costano di meno
      \\\hline
\end{longtable}

\pagebreak

\subsection*{Requisiti estratti dalla specifica e dall'interazione col cliente}
%
% SDL sta per Specifica Disambiguata Linea
%

\begin{longtable}{|p{0.8cm}|p{13.4cm}|p{2.1cm}|}
      \hline
      \rowcolor{tblhdrcolor}
      \multicolumn{1}{|c|}{\textbf{Id}}                                       &
      \multicolumn{1}{|c|}{\textbf{Descrizione}}                              &
      \multicolumn{1}{|c|}{\textbf{Origine}}
      \\\hline
      R1                                                                      &
      Ogni cinema ha un numero arbitrario di sale.                            &
      SDL 4, 5
      \\\hline
      R2                                                                      &
      Ogni cinema ha una fascia oraria di apertura.                           &
      SDL 26, 27
      \\\hline
      R3                                                                      &
      Ogni sala è identificata da un numero di sala.                          &
      SDL 5
      \\\hline
      R4                                                                      &
      Ogni sala ha un numero arbitrario di posti.                             &
      SDL 5
      \\\hline
      R5                                                                      &
      Ogni posto è identificato da una lettera per la fila ed un numero
      di posto.                                                               &
      SDL 6
      \\\hline
      R6                                                                      &
      In ogni sala si possono proiettare più film nello stesso giorno.        &
      SDL 8
      \\\hline
      R7                                                                      &
      Ogni film può essere proiettato più volte nello stesso giorno ed
      in sale differenti.                                                     &
      SDL 8, 9
      \\\hline
      R8                                                                      &
      Ogni film ha una durata, un nome, un cast e una casa cinematografica.   &
      SDL 10
      \\\hline
      R9                                                                      &
      Il prezzo di una prenotazione dipende dal film proiettato, dalla
      sala, dalla data e dall'orario in cui avviene la proiezione.            &
      I11, SDL 10, 11, 12
      \\\hline
      R10                                                                     &
      Gli amministratori gestiscono la programmazione delle proiezioni
      potendo:
      \begin{itemize}
            \item visualizzare le proiezioni in programma
            \item creare nuove proiezioni in programma
            \item modificare proiezioni esistenti in programma
            \item eliminare proiezioni esistenti in programma
            \item assegnare ad una proiezione in programma un proiezionista
      \end{itemize}         &
      I1
      \\\hline
      R11                                                                     &
      I clienti possono prenotare un posto per una proiezione completando
      la procedura di prenotazione.                                           &
      SDL 14, 15
      \\\hline
      R12                                                                     &
      I clienti hanno a disposizione 10 minuti per completare la procedura
      di prenotazione.                                                        &
      SDL 16
      \\\hline
      R13                                                                     &
      La procedura di prenotazione è formata dai seguenti passaggi:
      \begin{enumerate}
            \item il cliente seleziona un cinema
            \item il sistema permette al cliente di verificare
                  il palinsesto del cinema selezionato
            \item il cliente seleziona una proiezione
            \item il sistema permette al cliente di verificare
                  i posti disponibili
            \item il cliente seleziona un posto disponibile
            \item il cliente inserisce i dati relativi alla carta
                  di credito (numero, intestatario, data di scadenza,
                  codice CVV)
            \item il sistema richiede al servizio di pagamento di
                  generare la relativa transazione
            \item il servizio di pagamento restituisce al sistema
                  il codice della transazione generata
            \item il sistema genera la prenotazione e la registra
                  come prenotazione confermata
            \item il sistema restituisce al cliente un codice di
                  prenotazione
      \end{enumerate}                   &
      I3, SDL 16, 17, 18, 19
      \\\hline
      R14                                                                     &
      I clienti possono annullare la propria prenotazione completando la
      procedura di annullamento di una prenotazione.                          &
      SDL 20
      \\\hline
      R15                                                                     &
      I clienti possono annullare la propria prenotazione entro 30 minuti
      dall'inizio della proiezione.                                           &
      SDL 19, 20
      \\\hline
      R16                                                                     &
      La procedura di annullamento di una prenotazione è formata dai
      seguenti passaggi:
      \begin{enumerate}
            \item il cliente fornisce al sistema un codice di
                  prenotazione
            \item il sistema richiede al servizio di pagamento di
                  annullare la relativa transazione
            \item il sistema registra la prenotazione come prenotazione
                  annullata
      \end{enumerate}             &
      SDL 20, 21
      \\\hline
      R17                                                                     &
      I dipendenti sono divisi in maschere e proiezionisti.                   &
      SDL 23, 24
      \\\hline
      R18                                                                     &
      Ogni dipendente può effettuare un turno in un qualunque cinema.         &
      I10
      \\\hline
      R19                                                                     &
      Gli amministratori gestiscono la programmazione dei turni
      potendo:
      \begin{itemize}
            \item visualizzare i turni in programma
            \item creare nuovi turni in programma
            \item modificare turni esistenti in programma
            \item eliminare turni esistenti in programma
      \end{itemize}                           &
      SDL 24
      \\\hline
      R20                                                                     &
      La durata di una giornata lavorativa di un dipendente deve essere di
      massimo otto ore.                                                       &
      SDL 24
      \\\hline
      R21                                                                     &
      I turni sono gestiti su base settimanale.                               &
      SDL 24, 25
      \\\hline
      R22                                                                     &
      Gli amministratori possono generare un report che permette di sapere se:
      \begin{itemize}
            \item qualche proiezione è sprovvista di proiezionista
            \item qualche cinema, in qualche fascia oraria di apertura,
                  è sprovvisto di almeno due maschere
      \end{itemize}             &
      SDL 25, 26, 27
      \\\hline
      R23                                                                     &
      Le maschere validano le prenotazioni mediante il codice di prenotazione
      presente sui biglietti.                                                 &
      SDL 29, 30
      \\\hline
      R24                                                                     &
      Una prenotazione validata viene associata al fatto che quel determinato
      posto, per quella determinata proiezione, è stato occupato dal cliente. &
      SDL 30, 31
      \\\hline
      R25                                                                     &
      Non è possibile utilizzare più volte lo stesso codice di prenotazione
      per accedere al cinema.                                                 &
      SDL 32
      \\\hline
      R26                                                                     &
      Gli amministratori possono generare dei report mensili che mostrano
      per ogni cinema e per ogni sala:
      \begin{itemize}
            \item quante sono le prenotazioni confermate
            \item quante sono le prenotazioni annullate
            \item quante sono le prenotazioni scadute
      \end{itemize}                            &
      SDL 34, 35, 36, 37
      \\\hline
\end{longtable}

\pagebreak

\subsection*{Glossario dei Termini}
%
% Realizzare un dizionario dei termini, compilando la tabella qui sotto,
% a partire dalle specifiche precedentemente disambiguate
%
\begin{longtable}{|p{3.86cm}|p{3.86cm}|p{3.86cm}|p{3.86cm}|}
      \hline
      \rowcolor{tblhdrcolor}
      \multicolumn{1}{|c|}{\textbf{Termine}}
       & \multicolumn{1}{|c|}{\textbf{Descrizione}}
       & \multicolumn{1}{|c|}{\textbf{Sinonimi}}
       & \multicolumn{1}{|c|}{\textbf{Collegamenti}}
      \\\hline
      Cinema
       & Un cinema facente parte della catena cinematografica.
       &
       & Sala, Fascia oraria di apertura
      \\\hline
      Fascia oraria di apertura
       & Periodo temporale durante il quale il relativo cinema è aperto. [...]
       &
       &
      \\ \hline
      Sala
       & Una sala di un cinema in cui vengono effettuate proiezioni.
       &
       & Numero di sala, Posto
      \\\hline
      Numero di sala
       & Numero intero identificante una sala in un cinema.
       &
       &
      \\\hline
      Posto
       & Sedia, poltrona e sim. in una sala prenotabile da un cliente
      per una proiezione.
       &
       & Numero di posto
      \\\hline
      Numero di posto
       & Numero intero identificante la colonna del posto.
       &
       &
      \\\hline
      Film
       & Produzione cinematografica proiettabile in una sala di un cinema.
       &
       & Cast, Casa cinematografica
      \\\hline
      Cast
       & Insieme degli attori principali che lavora nella produzione di un film.
       & Cast degli attori protagonisti
       &
      \\\hline
      Casa cinematografica
       & Impresa che produce film.
       &
       &
      \\ \hline
      Proiezione
       & Riproduzione di un film in una sala in una certa data e in un
      certo orario.
       & Visione, Spettacolo
       & Sala, Film
      \\ \hline
      Programmazione delle proiezioni
       & La pianificazione delle proiezioni future.
       & Palinsesto
       &
      \\ \hline
      Palinsesto
       & Prospetto della programmazione delle proiezioni di un cinema.
       &
       & Programmazione
      \\ \hline
      Cliente
       & Utente che può effettuare la procedura di prenotazione.
       & Utente, Spettatore
       &
      \\ \hline
      Biglietto
       & Titolo di accesso ad un cinema, contiene i dati di una prenotazione
      ed è valido per la relativa proiezione.
       &
       & Prenotazione, Codice di prenotazione
      \\ \hline
      Prenotazione
       & Diritto a sedere su un certo posto per una certa proiezione acquisibile
      completando la procedura di prenotazione, il suo prezzo varia in base alla
      proiezione.
       & Biglietto
       & Cliente, Proiezione, Posto, Codice di Prenotazione, Carta di credito
      \\ \hline
      Codice di prenotazione
       & Codice univoco identificante una prenotazione.
      Permette di validare il relativo biglietto o di annullare la
      relativa prenotazione.
       &
       &
      \\ \hline
      Procedura di prenotazione
       & Processo tramite cui un cliente può effettuare una prenotazione.
       &
       & Cliente, Prenotazione, Posto disponibile, Codice di prenotazione,
      Carta di credito
      \\ \hline
      Posto disponibile
       & In riferimento ad una certa proiezione, posto non prenotato.
       &
       & Posto, Proiezione, Prenotazione
      \\ \hline
      Carta di credito
       & Strumento che consente all’intestatario di ottenere l'addebitamento
      del prezzo di beni e servizi con enti convenzionati con l’emittente.
       &
       &
      \\ \hline
      Biglietto utilizzato
       & Un biglietto associato ad una prenotazione validata.
       &
       & Biglietto, Codice di prenotazione
      \\ \hline
      Prenotazione confermata
       & Una prenotazione non annullata.
       &
       & Prenotazione
      \\ \hline
      Prenotazione annullata
       & Una prenotazione che è stata annullata dal cliente.
       &
       & Prenotazione
      \\ \hline
      Prenotazione validata
       & Una prenotazione confermata validata da una maschera.
       &
       & Prenotazione, Biglietto
      \\ \hline
      Prenotazione scaduta
       & Una prenotazione confermata relativa ad proiezione effettuata
      non validata da una maschera.
       &
       & Prenotazione, Biglietto
      \\ \hline
      Amministratore
       & Un amministratore della catena cinematografica.
       & Amministrazione della catena, Amministratori della catena
       &
      \\ \hline
      Dipendente
       & Un dipendente della catena cinematografica.
       &
       & Turno
      \\ \hline
      Maschera
       & Un tipo di dipendente.
       &
       &
      \\ \hline
      Proiezionista
       & Un tipo di dipendente.
       &
       &
      \\ \hline
      Turno
       & Un'intervallo di tempo di massimo otto ore all'interno di una giornata
      che definisce il periodo lavorativo di un dipendete.
       & Turno di lavoro
       &
      \\ \hline
      Programmazione dei turni
       & La pianificazione dei turni dei dipendenti svolta settimanalmente.
       & Palinsesto
       &
      \\ \hline
\end{longtable}

\pagebreak

\subsection*{Raggruppamento dei requisiti in insiemi omogenei}
%
% Per ciascun elemento "più importante" della specifica
% (riportata anche nel glossario precedente), estrapolare dalla specifica
% disambiguata le frasi ad esso associate.
% Compilare una tabella separata per ciascun elemento individuato.
%
\begin{tabularx}{\linewidth}{|X|}
      \hline
      \rowcolor{tblhdrcolor}
      \textbf{Frasi relative ai cinema} \\\hline
      Ogni cinema ha un numero arbitrario di sale.

      Ogni cinema ha una fascia oraria di apertura.

      Ogni dipendente può effettuare un turno in un qualunque cinema.

      Gli amministratori possono generare un report che permette di
      sapere se:
      \begin{itemize}
            \item qualche proiezione è sprovvista di proiezionista
            \item qualche cinema, in qualche fascia oraria di apertura,
                  è sprovvisto di almeno due maschere
      \end{itemize}
      \\ \hline
\end{tabularx}

\begin{tabularx}{\linewidth}{|X|}
      \hline
      \rowcolor{tblhdrcolor}
      \textbf{Frasi relative alle sale} \\\hline
      Ogni cinema ha un numero arbitrario di sale.

      Ogni sala è identificata da un numero di sala.

      Ogni sala ha un numero arbitrario di posti.

      In ogni sala si possono proiettare più film nello stesso giorno.

      Ogni film può essere proiettato più volte nello stesso giorno ed
      in sale differenti.

      Il prezzo di una prenotazione dipende dal film proiettato, dalla
      sala, dalla data e dall'orario in cui avviene la proiezione.
      \\ \hline
\end{tabularx}

\begin{tabularx}{\linewidth}{|X|}
      \hline
      \rowcolor{tblhdrcolor}
      \textbf{Frasi relative ai posti} \\\hline
      Ogni posto è identificato da una lettera per la fila ed un numero
      di posto.

      I clienti possono prenotare un posto per una proiezione completando la
      procedura di prenotazione.

      Una prenotazione validata viene associata al fatto che quel determinato
      posto, per quella determinata proiezione, è stato occupato dal cliente.
      \\ \hline
\end{tabularx}

\begin{tabularx}{\linewidth}{|X|}
      \hline
      \rowcolor{tblhdrcolor}
      \textbf{Frasi relative ai film} \\\hline
      In ogni sala si possono proiettare più film nello stesso giorno.

      Ogni film può essere proiettato più volte nello stesso giorno ed
      in sale differenti.

      Ogni film ha una durata, un nome, un cast e una
      casa cinematografica.

      Il prezzo di una prenotazione dipende dal film proiettato, dalla
      sala, dalla data e dall'orario in cui avviene la proiezione.
      \\ \hline
\end{tabularx}

\begin{tabularx}{\linewidth}{|X|}
      \hline
      \rowcolor{tblhdrcolor}
      \textbf{Frasi relative alle proiezioni} \\\hline
      In ogni sala si possono proiettare più film nello stesso giorno.

      Ogni film può essere proiettato più volte nello stesso giorno ed in sale
      differenti.

      Il prezzo di una prenotazione dipende dal film proiettato, dalla
      sala, dalla data e dall'orario in cui avviene la proiezione.

      I clienti possono prenotare un posto per una proiezione
      completando la procedura di prenotazione.

      Una prenotazione validata viene associata al fatto che quel
      determinato posto, per quella determinata proiezione,
      è stato occupato dal cliente.
      \\ \hline
\end{tabularx}

\begin{tabularx}{\linewidth}{|X|}
      \hline
      \rowcolor{tblhdrcolor}
      \textbf{Frasi relative alla programmazione delle proiezioni} \\\hline

      In ogni sala si possono proiettare più film nello stesso giorno.

      Ogni film può essere proiettato più volte nello stesso giorno ed in sale
      differenti.

      Il prezzo di una prenotazione dipende dal film proiettato, dalla
      sala, dalla data e dall'orario in cui avviene la proiezione.

      Gli amministratori gestiscono la programmazione delle proiezioni.
      \\ \hline
\end{tabularx}

\begin{tabularx}{\linewidth}{|X|}
      \hline
      \rowcolor{tblhdrcolor}
      \textbf{Frasi relative alle prenotazioni} \\\hline
      Il prezzo di una prenotazione dipende dal film proiettato, dalla
      sala, dalla data e dall'orario in cui avviene la proiezione.

      I clienti possono prenotare un posto per una proiezione
      completando la procedura di prenotazione.

      Le maschere validano le prenotazioni mediante il codice
      di prenotazione presente sui biglietti.

      Una prenotazione validata viene associata al fatto che quel
      determinato posto, per quella determinata proiezione,
      è stato occupato dal cliente.

      Non è possibile utilizzare più volte lo stesso codice di
      prenotazione per accedere al cinema.

      Gli amministratori possono generare dei report mensili che
      mostrano per ogni cinema e per ogni sala:
      \begin{itemize}
            \item quante sono le prenotazioni confermate
            \item quante sono le prenotazioni annullate
            \item quante sono le prenotazioni scadute
      \end{itemize}
      \\ \hline
\end{tabularx}

\begin{tabularx}{\linewidth}{|X|}
      \hline
      \rowcolor{tblhdrcolor}
      \textbf{Frasi relative a procedura di prenotazione} \\\hline
      I clienti possono prenotare un posto per una proiezione
      completando la procedura di prenotazione.

      I clienti hanno a disposizione 10 minuti per completare la
      procedura di prenotazione.

      La procedura di prenotazione è formata dai seguenti passaggi:
      \begin{enumerate}
            \item il cliente seleziona un cinema
            \item il sistema permette al cliente di verificare
                  il palinsesto del cinema selezionato
            \item il cliente seleziona una proiezione
            \item il cliente seleziona un posto disponibile
            \item il cliente inserire i dati relativi alla carta
                  di credito (numero, intestatario, data di scadenza,
                  codice CVV)
            \item il sistema restituisce al cliente un codice di
                  prenotazione
      \end{enumerate}

      I clienti possono annullare la propria prenotazione usando il
      codice di prenotazione entro 30 minuti dall'inizio della
      proiezione.
      \\ \hline
\end{tabularx}

\begin{tabularx}{\linewidth}{|X|}
      \hline
      \rowcolor{tblhdrcolor}
      \textbf{Frasi relative agli amministratori} \\\hline
      Gli amministratori gestiscono la programmazione delle proiezioni.

      Gli amministratori gestiscono la programmazione dei turni.

      Gli amministratori possono generare un report che permette di
      sapere se:
      \begin{itemize}
            \item qualche proiezione è sprovvista di proiezionista
            \item qualche cinema, in qualche fascia oraria di apertura,
                  è sprovvisto di almeno due maschere
      \end{itemize}

      Gli amministratori possono generare dei report mensili che
      mostrano per ogni cinema e per ogni sala:
      \begin{itemize}
            \item quante sono le prenotazioni confermate
            \item quante sono le prenotazioni annullate
            \item quante sono le prenotazioni scadute
      \end{itemize}
      \\ \hline
\end{tabularx}

\begin{tabularx}{\linewidth}{|X|}
      \hline
      \rowcolor{tblhdrcolor}
      \textbf{Frasi relative ai dipendenti} \\\hline
      I dipendenti sono divisi in maschere e proiezionisti.

      Ogni dipendente può effettuare un turno in un qualunque cinema.
      \\ \hline
\end{tabularx}

\begin{tabularx}{\linewidth}{|X|}
      \hline
      \rowcolor{tblhdrcolor}
      \textbf{Frasi relative alle maschere} \\\hline
      I dipendenti sono divisi in maschere e proiezionisti.

      Le maschere validano le prenotazioni mediante il codice
      di prenotazione presente sui biglietti.

      Gli amministratori possono generare un report che permette di
      sapere se:
      \begin{itemize}
            \item qualche proiezione è sprovvista di proiezionista
            \item qualche cinema, in qualche fascia oraria di apertura,
                  è sprovvisto di almeno due maschere
      \end{itemize}
      \\ \hline
\end{tabularx}

\begin{tabularx}{\linewidth}{|X|}
      \hline
      \rowcolor{tblhdrcolor}
      \textbf{Frasi relative ai proiezionisti} \\\hline
      I dipendenti sono divisi in maschere e proiezionisti.

      Gli amministratori possono generare un report che permette di
      sapere se:
      \begin{itemize}
            \item qualche proiezione è sprovvista di proiezionista
            \item qualche cinema, in qualche fascia oraria di apertura,
                  è sprovvisto di almeno due maschere
      \end{itemize}
      \\ \hline
\end{tabularx}

\begin{tabularx}{\linewidth}{|X|}
      \hline
      \rowcolor{tblhdrcolor}
      \textbf{Frasi relative ai turni} \\\hline
      Ogni dipendente può effettuare un turno in un qualunque cinema.

      I turni devono essere di massimo otto ore.

      I turni sono gestiti su base settimanale.
      \\ \hline
\end{tabularx}

\begin{tabularx}{\linewidth}{|X|}
      \hline
      \rowcolor{tblhdrcolor}
      \textbf{Frasi relative alla programmazione dei turni} \\\hline
      Gli amministratori gestiscono la programmazione dei turni.

      I turni sono gestiti su base settimanale.
      \\ \hline
\end{tabularx}
