\section{Analisi dei Requisiti}
%
% Lo scopo di questa sezione è raffinare la specifica fornita, andando ad
% effettuare un'operazione preliminare di disambiguazione.
%
\subsection*{Identificazione dei termini ambigui e correzioni possibili}
%
% Compilare la seguente tabella, facendo riferimento alla specifica del
% minimondo di riferimento precedentemente indicata.
% Individuare i termini ambigui nella specifica
% (indicando la linea in cui essi compaiono), indicare il nuovo termine che
% si intende adottare nella specifica, ed indicare il motivo del cambiamento
% che si propone.
%
\begin{tabularx}{\linewidth}{|p{1.5cm}|p{3cm}|p{3cm}|X|}
    \hline
    \rowcolor{tblhdrcolor}
    \multicolumn{1}{|c|}{\textbf{Linea}}
     & \multicolumn{1}{|c|}{\textbf{Termine}}
     & \multicolumn{1}{|c|}{\textbf{Nuovo termine}}
     & \multicolumn{1}{|c|}{\textbf{Motivo correzione}}
    \\\hline
    4
     & amministrazione della catena
     & amministratori
     & Il termine "amministrazione" viene utilizzato come sinonimo di
    "amministratori" generando un'ambiguità.

    I termini "della catena" sono pleonastici.
    \\ \hline
    24, 25
     & amministratori della catena
     & amministratori
     & I termini "della catena" sono pleonastici.
    \\ \hline
    14, 19
     & utente
     & cliente
     & Il termine "utente" viene utilizzato come sinonimo di
    "cliente" generando un'ambiguità.
    \\ \hline
    32
     & spettatore
     & cliente
     & Il termine "spettatore" viene utilizzato come sinonimo di
    "cliente" generando un'ambiguità.
    \\ \hline
    14
     & visione
     & proiezione
     & Il termine "visione" viene utilizzato come sinonimo di
    "proiezione" generando un'ambiguità.
    \\ \hline
    26
     & spettacolo
     & proiezione
     & Il termine "spettacolo" viene utilizzato come sinonimo di
    "proiezione" generando un'ambiguità.
    \\ \hline
    16
     & perfezionare
     & completare
     & Il termine "perfezionare" è utilizzato come sinonimo di "completare"
    rendendo più difficile la comprensione del testo.
    \\ \hline
    10
     & è associato al
     & [ha] un
     & Il termine "è associato al" ha un significato omologo a
    "[ha] un" rendendo più difficile la comprensione del testo.
    \\ \hline
\end{tabularx}

\pagebreak
\subsubsection*{Interlocuzione con il cliente}

\begin{tabularx}{\linewidth}{|X|X|}
    \hline
    \rowcolor{tblhdrcolor}
    \multicolumn{1}{|c|}{\textbf{Messaggio}}
     & \multicolumn{1}{|c|}{\textbf{Risposta}}
    \\\hline
    Non è chiaro il significato di questo periodo nella specifica:

    "L'amministrazione della catena gestisce i cinema."

    La frase in questione è circostanziale o con il termine "gestisce" si vuole
    far riferimento a delle operazioni specifiche sui cinema che non sono
    state definite?
     & \hfill
    \\\hline
    La specifica presenta un'ambiguità:

    "Ciascun cinema ha un numero arbitrario di sale,
    identificato da un numero di sala."

    Dal genere del termine "identificato" si evince che il soggetto della
    subordinata può essere il numero o il cinema.

    Dalle mie conoscenze del dominio sospetto che si intenda tuttavia
    che sia la sala ad essere identificata dal numero di sala,
    cosa si intende?
     & \hfill
    \\\hline
    La specifica presenta un'ambiguità:

    "In ogni sala vengono proiettati più film quotidianamente."

    Si vuole affermare che
    "in ogni sala devono essere proiettati almeno due film ogni giorno"
    o
    "in ogni sala si possono proiettare più film nello stesso giorno"?
     & \hfill
    \\\hline
    La specifica presenta un'ambiguità:

    "Dopo aver scelto il posto, al cliente è data la possibilità di inserire
    i dati relativi alla propria carta di credito [...] Una volta inseriti
    questi dati, il sistema restituisce all’utente un codice di prenotazione."

    Con questa frase si vuole affermare che l'inserimento dei dati della carta
    di credito è un passaggio opzionale, ed è quindi concesso al cliente
    di pagare la prenotazione alla cassa successivamente,
    o è invece obbligatorio l'inserimento dei dati della carta di credito per
    completare la procedura?
     & \hfill
    \\\hline
    Non è chiaro il significato di questo periodo nella specifica:

    "Gli amministratori della catena definiscono i turni di lavoro[...]"

    Quali operazioni specifiche sui turni di lavoro si richiede di mettere
    a disposizione degli amministratori?
     & \hfill
    \\\hline
\end{tabularx}
\pagebreak

\subsubsection*{Specifica disambiguata}

\begin{tabularx}{\linewidth}{|X|}
    \hline
    % Riportare in questo riquadro la specifica di progetto corretta,
    % applicando le disambiguazioni proposte.
    \begin{itemize}
        \item Ogni cinema ha un numero arbitrario di sale.
        \item Ogni sala è identificata da un numero di sala.
        \item Ogni sala ha un numero arbitrario di posti.
        \item Ogni posto è identificato da una lettera per la fila ed un numero.
        \item In ogni sala si possono proiettare più film nello stesso giorno.
        \item Ogni film può essere proiettato più volte nello stesso giorno ed
              in sale differenti.
        \item Ogni film ha una durata, un nome, un cast degli attori
              protagonisti e una casa cinematografica.
        \item Il costo del biglietto per una proiezione dipende dal film
              proiettato, dalla sala e dall'orario in cui avviene la proiezione.
        \item I clienti possono prenotare un posto per una proiezione
              completando la procedura di prenotazione.
        \item I clienti hanno a disposizione 10 minuti per completare la
              procedura di prenotazione.
        \item La procedura di prenotazione è composta dai seguenti passaggi:
              \begin{enumerate}
                  \item il cliente deve scegliere un posto disponibile
                  \item il cliente deve inserire i dati relativi alla carta
                        di credito (numero, intestatario, data di scadenza,
                        codice CVV)
                  \item il sistema deve restituire al cliente un codice di
                        prenotazione
              \end{enumerate}
        \item I clienti possono annullare la sua prenotazione usando il codice
              di prenotazione entro 30 minuti dall'inizio della proiezione.
        \item I dipendenti sono divisi in maschere e proiezionisti.
        \item Gli amministratori definiscono i turni di lavoro.
        \item I turni di lavoro devono essere di massimo otto ore.
        \item I turni sono gestiti su base settimanale.
        \item Gli amministratori possono generare un report che permette di
              sapere se qualche proiezione è sprovvista di proiezionista o se
              qualche cinema, in qualche fascia oraria di apertura, è sprovvisto
              di almeno due maschere.
        \item Le maschere verificano i biglietti all'ingresso mediante il codice
              di prenotazione.
        \item Un biglietto utilizzato viene associato al fatto che quel
              determinato posto, per quella determinata proiezione, è stato
              occupato dal cliente.
        \item Non è possibile utilizzare più volte lo stesso codice di
              prenotazione per accedere al cinema.
        \item Gli amministratori possono generare dei report mensili che
              mostrano per ciascun cinema e ciascuna sala quante prenotazioni
              sono state confermate, quante sono state annullate, e quante
              prenotazioni confermate non sono state utilizzate.
    \end{itemize}
    \\\hline
\end{tabularx}

\subsection*{Glossario dei Termini}
%
% Realizzare un dizionario dei termini, compilando la tabella qui sotto,
% a partire dalle specifiche precedentemente disambiguate
%
\begin{tabularx}{\linewidth}{|X|X|X|X|}
    \hline
    \rowcolor{tblhdrcolor}
    \multicolumn{1}{|c|}{\textbf{Termine}}
     & \multicolumn{1}{|c|}{\textbf{Descrizione}}
     & \multicolumn{1}{|c|}{\textbf{Sinonimi}}
     & \multicolumn{1}{|c|}{\textbf{Collegamenti}}
    \\\hline
    \hfill
     & \hfill
     & \hfill
     & \hfill
    \\ \hline
\end{tabularx}

\subsection*{Raggruppamento dei requisiti in insiemi omogenei}
%
% Per ciascun elemento "più importante" della specifica
% (riportata anche nel glossario precedente), estrapolare dalla specifica
% disambiguata le frasi ad esso associate.
% Compilare una tabella separata per ciascun elemento individuato.
%
\begin{tabularx}{\linewidth}{|X|}
    \hline
    \rowcolor{tblhdrcolor}
    \textbf{Frasi relative a ...} \\\hline
    %
    \\ \hline
\end{tabularx}
