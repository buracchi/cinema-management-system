\section{Analisi dei Requisiti}

\begin{templateblock}
    Lo scopo di questa sezione è raffinare la specifica fornita, andando ad
    effettuare un’operazione preliminare di disambiguazione.
\end{templateblock}

\subsection*{Identificazione dei termini ambigui e correzioni possibili}

\begin{templateblock}
    Compilare la seguente tabella, facendo riferimento alla specifica del
    minimondo di riferimento precedentemente indicata.
    Individuare i termini ambigui nella specifica
    (indicando la linea in cui essi compaiono), indicare il nuovo termine che
    si intende adottare nella specifica, ed indicare il motivo del cambiamento
    che si propone.
\end{templateblock}

\begin{tabularx}{\linewidth}{|l|l|l|X|}
    \hline
    \rowcolor{tblhdrcolor}
    \multicolumn{1}{|c|}{\textbf{Linea}}
     & \multicolumn{1}{|c|}{\textbf{Termine}}
     & \multicolumn{1}{|c|}{\textbf{Nuovo termine}}
     & \multicolumn{1}{|c|}{\textbf{Motivo correzione}}
    \\\hline
    \hfill
     & \hfill
     & \hfill
     & \hfill
    \\ \hline
\end{tabularx}

\subsubsection*{Specifica disambiguata}

\begin{tabularx}{\linewidth}{|X|}
    \hline
    \begin{templateblock}
        Riportare in questo riquadro la specifica di progetto corretta,
        applicando le disambiguazioni proposte.
    \end{templateblock}
    \\\hline
\end{tabularx}

\subsection*{Glossario dei Termini}

\begin{templateblock}
    Realizzare un dizionario dei termini, compilando la tabella qui sotto,
    a partire dalle specifiche precedentemente disambiguate
\end{templateblock}

\begin{tabularx}{\linewidth}{|X|X|X|X|}
    \hline
    \rowcolor{tblhdrcolor}
    \multicolumn{1}{|c|}{\textbf{Termine}}
     & \multicolumn{1}{|c|}{\textbf{Descrizione}}
     & \multicolumn{1}{|c|}{\textbf{Sinonimi}}
     & \multicolumn{1}{|c|}{\textbf{Collegamenti}}
    \\\hline
    \hfill
     & \hfill
     & \hfill
     & \hfill
    \\ \hline
\end{tabularx}

\subsection*{Raggruppamento dei requisiti in insiemi omogenei}

\begin{templateblock}
    Per ciascun elemento “più importante” della specifica
    (riportata anche nel glossario precedente), estrapolare dalla specifica
    disambiguata le frasi ad esso associate.
    Compilare una tabella separata per ciascun elemento individuato.
\end{templateblock}

\begin{tabularx}{\linewidth}{|X|}
    \hline
    \rowcolor{tblhdrcolor}
    \textbf{Frasi relative a ...} \\\hline
    %
    \\ \hline
\end{tabularx}
