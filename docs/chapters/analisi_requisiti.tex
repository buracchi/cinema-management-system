\section{Analisi dei Requisiti}
%
% Lo scopo di questa sezione è raffinare la specifica fornita, andando ad
% effettuare un'operazione preliminare di disambiguazione.
%
\subsection*{Identificazione dei termini ambigui e correzioni possibili}
%
% Compilare la seguente tabella, facendo riferimento alla specifica del
% minimondo di riferimento precedentemente indicata.
% Individuare i termini ambigui nella specifica
% (indicando la linea in cui essi compaiono), indicare il nuovo termine che
% si intende adottare nella specifica, ed indicare il motivo del cambiamento
% che si propone.
%
\begin{tabularx}{\linewidth}{|p{1.5cm}|p{3cm}|p{3cm}|X|}
    \hline
    \rowcolor{tblhdrcolor}
    \multicolumn{1}{|c|}{\textbf{Linea}}
     & \multicolumn{1}{|c|}{\textbf{Termine}}
     & \multicolumn{1}{|c|}{\textbf{Nuovo termine}}
     & \multicolumn{1}{|c|}{\textbf{Motivo correzione}}
    \\\hline
    1, 23
     & catena di cinema
     & catena cinematografica
     & Il termine "cinema" viene utilizzato per descrivere un’entità nel
    dominio di riferimento, il suo utilizzo nel descrivere un elemento
    differente crea pertanto un’ambiguità.
    \\ \hline
    4
     & amministrazione della catena
     & amministratori
     & Il termine "amministrazione" viene utilizzato come sinonimo di
    "amministratori" generando un'ambiguità.

    I termini "della catena" sono pleonastici.
    \\ \hline
    24, 25
     & amministratori della catena
     & amministratori
     & I termini "della catena" sono pleonastici.
    \\ \hline
    14, 19
     & utente
     & cliente
     & Il termine "utente" viene utilizzato come sinonimo di
    "cliente" generando un'ambiguità.
    \\ \hline
    32
     & spettatore
     & cliente
     & Il termine "spettatore" viene utilizzato come sinonimo di
    "cliente" generando un'ambiguità.
    \\ \hline
    14
     & visione
     & proiezione
     & Il termine "visione" viene utilizzato come sinonimo di
    "proiezione" generando un'ambiguità.
    \\ \hline
    26
     & spettacolo
     & proiezione
     & Il termine "spettacolo" viene utilizzato come sinonimo di
    "proiezione" generando un'ambiguità.
    \\ \hline
    24
     & turni di lavoro
     & turni
     & I termini "di lavoro" sono pleonastici.
    \\ \hline
    10
     & cast degli attori protagonisti
     & cast
     & Utilizziamo per semplicità il termine "cast" per riferirsi al
    "cast degli attori protagonisti".
    \\ \hline
    16
     & perfezionare
     & completare
     & Il termine "perfezionare" è utilizzato come sinonimo di "completare"
    rendendo più difficile la comprensione del testo.
    \\ \hline
    27
     & verifica
     & validazione
     & Il termine "verifica" è utilizzato come sinonimo di "validazione"
    generando un'ambiguità.
    \\ \hline
    10
     & è associato al
     & [ha] un
     & Il termine "è associato al" ha un significato omologo a
    "[ha] un" ma rende più difficile la comprensione del testo.
    \\ \hline
\end{tabularx}

\pagebreak
\subsubsection*{Interlocuzione con il cliente}

\begin{longtable}{|p{8.15cm}|p{8.15cm}|}
    \hline
    \rowcolor{tblhdrcolor}
    \multicolumn{1}{|c|}{\textbf{Messaggio}}
     & \multicolumn{1}{|c|}{\textbf{Risposta}}
    \\\hline
    Non è chiaro il significato di questo periodo nella specifica:

    "L'amministrazione della catena gestisce i cinema."

    La frase in questione è circostanziale o con il termine "gestisce" si vuole
    far riferimento a delle operazioni specifiche sui cinema che non sono state
    definite e che si richiede di mettere a disposizione degli amministratori?
     & Definisce, ad esempio, chi è che può decidere il palinsesto dei cinema,
    ossia l'amministrazione della catena può cambiare ad esempio la
    programmazione

    Così come anche i turni di lavoro dei dipendenti
    \\\hline
    Quando gli amministratori modificano la programmazione delle proiezioni
    dovrebbe essere il sistema ad aggiornare coerentemente tutti i palinsesti
    o la modifica manuale dei palinsesti è un'operazione che si richiede
    di mettere a disposizione degli amministratori?
     & Non ho capito sinceramente in cosa potrebbe consistere l'automatismo...
    \\\hline
    Attraverso la procedura di aggiornamento della programmazione delle
    proiezioni gli amministratori possono aggiornare le proiezioni in programma.

    Durante la procedura di prenotazione il cliente ha la possibilità di
    verificare il palinsesto di un cinema per decidere quale proiezione
    prenotare.

    Queste due informazioni sono corrette?
     & Si
    \\\hline
    Il palinsesto di un cinema mostrato al cliente durante la procedura di
    prenotazione dev'essere generato dal sistema o dev'essere possibile per
    gli amministratori modificarlo manualmente?
     & Credo ci sia un equivoco: come può un sistema generare automaticamente
    qualcosa se non ha le informazioni per farlo? Il palinsesto è solo la
    lista dei film con gli orari. Se gli amministratori non inseriscono i
    film con gli orari, cosa può generare il sistema in automatico?
    \\\hline
    Nel caso in cui il sistema non contenga le informazioni necessarie per
    generare il palinsesto il sistema informerà il cliente che non sono
    presenti proiezioni in programma.
    Durante la procedura di aggiornamento della programmazione delle
    proiezioni gli amministratori dovranno inserire nel sistema i dati relativi
    alle proiezioni, che comprendono i film e gli orari di proiezione.
    Queste informazioni risiederanno dunque nel sistema al termine della prima
    procedura di aggiornamento della programmazione delle proiezioni.
     & Continuo a non capire qual è il punto di tutta questa discussione...
    U+1F601

    Se non ci sono film, i film prenotabili sono zero, non vedo cosa cosa ci
    sia di diverso nell'"informare il cliente" di questo dal mostrare i film
    che può prenotare U+1F914

    In ultimo sì: sono gli amministratori che possono decidere la
    programmazione dei film, come dicevamo già prima!
    \\\hline
    Non è chiaro il significato di questo periodo nella specifica:

    "Gli amministratori della catena definiscono i turni di lavoro[...]"

    Quali operazioni specifiche sui turni di lavoro si richiede di mettere
    a disposizione degli amministratori?
     & \hfill
    \\\hline
\end{longtable}

\subsubsection*{Specifica disambiguata}

\begin{tabularx}{\linewidth}{|X|}
    \hline
    % Riportare in questo riquadro la specifica di progetto corretta,
    % applicando le disambiguazioni proposte.
    \begin{itemize}
        \item Gli amministratori possono modificare la programmazione delle
              proiezioni.
        \item Gli amministratori possono modificare la programmazione dei turni.
        \item Ogni cinema ha un numero arbitrario di sale.
        \item Ogni sala è identificata da un numero di sala.
        \item Ogni sala ha un numero arbitrario di posti.
        \item Ogni posto è identificato da una lettera per la fila ed un numero
              di posto.
        \item In ogni sala si possono proiettare più film nello stesso giorno.
        \item Ogni film può essere proiettato più volte nello stesso giorno ed
              in sale differenti.
        \item Ogni film ha una durata, un nome, un cast e una
              casa cinematografica.
        \item Il costo del biglietto per una proiezione dipende dal film
              proiettato, dalla sala e dall'orario in cui avviene la proiezione.
        \item I clienti possono prenotare un posto per una proiezione
              completando la procedura di prenotazione.
        \item I clienti hanno a disposizione 10 minuti per completare la
              procedura di prenotazione.
        \item La procedura di prenotazione è formata dai seguenti passaggi:
              \begin{enumerate}
                  \item il cliente seleziona un cinema
                  \item il sistema permette al cliente di verificare
                        il palinsesto del cinema selezionato
                  \item il cliente seleziona una proiezione
                  \item il cliente seleziona un posto disponibile
                  \item il cliente inserire i dati relativi alla carta
                        di credito (numero, intestatario, data di scadenza,
                        codice CVV)
                  \item il sistema restituisce al cliente un codice di
                        prenotazione
              \end{enumerate}
        \item I clienti possono annullare la propria prenotazione usando il
              codice di prenotazione entro 30 minuti dall'inizio della
              proiezione.
        \item I dipendenti sono divisi in maschere e proiezionisti.
        \item Gli amministratori definiscono i turni. (?)
        \item I turni devono essere di massimo otto ore.
        \item I turni sono gestiti su base settimanale.
        \item Gli amministratori possono generare un report che permette di
              sapere se qualche proiezione è sprovvista di proiezionista o se
              qualche cinema, in qualche fascia oraria di apertura, è sprovvisto
              di almeno due maschere.
        \item Le maschere validano i biglietti mediante il codice
              di prenotazione.
        \item Un biglietto utilizzato viene associato al fatto che quel
              determinato posto, per quella determinata proiezione, è stato
              occupato dal cliente.
        \item Non è possibile utilizzare più volte lo stesso codice di
              prenotazione per accedere al cinema.
        \item Gli amministratori possono generare dei report mensili che
              mostrano per ogni cinema e per ogni sala quante prenotazioni
              sono state confermate, quante sono state annullate, e quante
              prenotazioni confermate non sono state utilizzate.
    \end{itemize}
    \\\hline
\end{tabularx}

\pagebreak

\subsection*{Glossario dei Termini}
%
% Realizzare un dizionario dei termini, compilando la tabella qui sotto,
% a partire dalle specifiche precedentemente disambiguate
%
\begin{longtable}{|p{3.86cm}|p{3.86cm}|p{3.86cm}|p{3.86cm}|}
    \hline
    \rowcolor{tblhdrcolor}
    \multicolumn{1}{|c|}{\textbf{Termine}}
     & \multicolumn{1}{|c|}{\textbf{Descrizione}}
     & \multicolumn{1}{|c|}{\textbf{Sinonimi}}
     & \multicolumn{1}{|c|}{\textbf{Collegamenti}}
    \\\hline
    Cinema
     & Un cinema facente parte della catena cinematografica.
     &
     & Sala, Fascia oraria di apertura
    \\\hline
    Sala
     & Una sala di un cinema in cui vengono effettuate proiezioni.
     &
     & Numero di sala, Posto
    \\\hline
    Numero di sala
     & Numero intero identificante una sala in un cinema.
     &
     &
    \\\hline
    Posto
     & Sedia, poltrona e sim. in una sala su cui un cliente acquisisce
    il diritto di sedere acquistando il relativo biglietto.
     &
     & Numero di posto
    \\\hline
    Numero di posto
     & Numero intero identificante la colonna del posto.
     &
     &
    \\\hline
    Film
     & Produzione cinematografica proiettabile in una sala di un cinema.
     &
     & Cast, Casa cinematografica
    \\\hline
    Cast
     & Insieme degli attori principali che lavora nella produzione di un film.
     & Cast degli attori protagonisti
     &
    \\\hline
    Casa cinematografica
     & Impresa che produce film.
     &
     &
    \\ \hline
    Biglietto
     & Titolo di accesso per una proiezione rilasciato al termine di una
    procedura di prenotazione al cliente, ha un prezzo.
     &
     & Codice di prenotazione
    \\ \hline
    Proiezione
     & Trasmissione di un Film in una sala in un certo orario.
     & Visione, Spettacolo
     & Sala, Film
    \\ \hline
    Cliente
     & Persona che può acquistare un biglietto completando la procedura di
    prenotazione.
     & Utente, Spettatore
     &
    \\ \hline
    Prenotazione
     & Diritto a sedere sul posto prenotato per una certa proiezione di un
    cliente che ha completato la procedura di prenotazione.
     &
     & Cliente, Proiezione, Posto, Codice di Prenotazione, Carta di credito
    \\ \hline
    Procedura di prenotazione
     & Insieme di azioni che costituiscono il processo in cui un cliente può
    effettuare una prenotazione.
     &
     & Cliente, Prenotazione, Posto disponibile, Codice di prenotazione,
    Carta di credito
    \\ \hline
    Posto disponibile
     & Posto non prenotato durante una certa proiezione.
     &
     & Posto, Proiezione, Prenotazione
    \\ \hline
    Carta di credito
     & Strumento che consente all’intestatario di ottenere l'addebitamento
    del prezzo di beni e servizi con enti convenzionati con l’emittente.
     &
     &
    \\ \hline
    Codice di prenotazione
     & Codice univoco rilasciato al termine della procedura di prenotazione
    che permette di validare il relativo biglietto o di annullare la
    relativa prenotazione.
     &
     &
    \\ \hline
    Dipendente
     & Un dipendente della catena cinematografica.
     &
     & Turno
    \\ \hline
    Maschera
     & Un tipo di dipendente.
     &
     &
    \\ \hline
    Proiezionista
     & Un tipo di dipendente.
     &
     &
    \\ \hline
    Amministratore
     & Un amministratore della catena cinematografica.
     & Amministrazione della catena, Amministratori della catena
     &
    \\ \hline
    Turno
     & Un turno lavorativo di un dipendete.
     & Turno di lavoro
     &
    \\ \hline
    Fascia oraria di apertura
     & Periodo temporale (?) durante il quale il relativo cinema è aperto.
     &
     &
    \\ \hline
    Biglietto utilizzato
     & Un biglietto validato da un maschera.
     &
     & Biglietto, Codice di prenotazione
    \\ \hline
    Prenotazione confermata
     & Una prenotazione il cui codice di prenotazione corrisponde a quello di
    un biglietto utilizzato.
     &
     & Prenotazione, Biglietto utilizzato
    \\ \hline
    Prenotazione annullata
     & Una prenotazione che è stata annullata dal cliente.
     &
     & Prenotazione
    \\ \hline
    Prenotazione confermata non utilizzata
     & Una prenotazione il cui codice di prenotazione corrisponde a quello di
    un biglietto non utilizzato.
     &
     & Prenotazione, Biglietto
    \\ \hline
    Programmazione delle proiezioni
     & La pianificazione delle proiezioni future.
     & Palinsesto
     &
    \\ \hline
    Palinsesto
     & Prospetto della programmazione delle proiezioni di un cinema,
    aggiornato periodicamente.
    (?)
     &
     & Programmazione
    \\ \hline
\end{longtable}

\subsection*{Raggruppamento dei requisiti in insiemi omogenei}
%
% Per ciascun elemento "più importante" della specifica
% (riportata anche nel glossario precedente), estrapolare dalla specifica
% disambiguata le frasi ad esso associate.
% Compilare una tabella separata per ciascun elemento individuato.
%
\begin{tabularx}{\linewidth}{|X|}
    \hline
    \rowcolor{tblhdrcolor}
    \textbf{Frasi relative a ...} \\\hline
    %
    \\ \hline
\end{tabularx}
