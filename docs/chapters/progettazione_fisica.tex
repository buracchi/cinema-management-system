\section{Progettazione fisica}

\subsection*{Utenti e privilegi}
%
% Descrivere, all’interno dell’applicazione, quali utenti sono stati previsti
% con quali privilegi di accesso su quali tabelle, giustificando le scelte
% progettuali.
%
\subsection*{Strutture di memorizzazione}
%
% Compilare la tabella seguente indicando quali tipi di dato vengono
% utilizzati per memorizzare le informazioni di interesse nelle tabelle,
% per ciascuna tabella.
%
\begin{tabularx}{\linewidth}{|X|X|X|}
    \hline
    \multicolumn{3}{|>{\columncolor{tblhdrcolor}}l|}{
    \textbf{Tabella \textlangle{}nome\textrangle{}}} \\\hline
    \rowcolor{tblhdrcolor}
    \multicolumn{1}{|c|}{\textbf{Colonna}}
     & \multicolumn{1}{|c|}{\textbf{Tipo di dato}}
     & \multicolumn{1}{|c|}{\textbf{Attributi}
        \footnote{PK = primary key, NN = not null, UQ = unique,
            UN = unsigned, AI = auto increment.
            È ovviamente possibile specificare più di un attributo per
            ciascuna colonna.}}
    \\\hline
    \hfill
     & \hfill
     & \hfill
    \\ \hline
\end{tabularx}

\subsection*{Indici}
%
% Compilare la seguente tabella, per ciascuna tabella del database in cui
% sono presenti degli indici.
% Descrivere le motivazioni che hanno portato alla creazione di un indice.
%
\begin{tabularx}{\linewidth}{|X|X|}
    \hline
    \multicolumn{2}{|>{\columncolor{tblhdrcolor}}l|}{
    \textbf{Tabella \textlangle{}nome\textrangle{}}} \\\hline
    \rowcolor{tblhdrcolor}
    \multicolumn{1}{|l|}{\textbf{Indice \textlangle{}nome\textrangle{}}}
     & \multicolumn{1}{|l|}{\textbf{Tipo}
        \footnote{IDX = index, UQ = unique, FT = full text, PR = primary.}
        \textbf{:}}
    \\\hline
    Colonna 1
     & \textlangle{}nome\textrangle{}
    \\ \hline
\end{tabularx}

\subsection*{Trigger}
%
% Descrivere quali trigger sono stati implementati, mostrando il codice SQL
% per la loro istanziazione. Si faccia riferimento al fatto che il DBMS di
% riferimento richiede di utilizzare trigger anche per realizzare vincoli di
% check ed asserzioni.
%
\subsection*{Eventi}
%
% Descrivere quali eventi sono stati implementati, mostrando il codice SQL
% per la loro istanziazione. Si descriva anche se gli eventi sono
% istanziati soltanto in fase di configurazione del sistema, o se alcuni
% eventi specifici vengono istanziati in maniera effimera durante
% l’esecuzione di alcune procedure.
%
\subsection*{Viste}
%
% Mostrare e commentare il codice SQL necessario a creare tutte le viste
% necessarie per l’implementazione dell’applicazione.
%
\subsection*{Stored Procedures e transazioni}
%
% Mostrare e commentare le stored procedure che sono state realizzate per
% implementare la logica applicativa delle operazioni sui dati, evidenziando
% quando (e perché) sono state realizzate operazioni transazionali complesse.
%
