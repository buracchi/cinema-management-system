\section{Progettazione fisica}

\subsection*{Utenti e privilegi}

\begin{templateblock}
    Descrivere, all’interno dell’applicazione, quali utenti sono stati previsti
    con quali privilegi di accesso su quali tabelle, giustificando le scelte
    progettuali.
\end{templateblock}

\subsection*{Strutture di memorizzazione}

\begin{templateblock}
    Compilare la tabella seguente indicando quali tipi di dato vengono
    utilizzati per memorizzare le informazioni di interesse nelle tabelle,
    per ciascuna tabella.
\end{templateblock}

\begin{tabularx}{\linewidth}{|Y|Y|Y|}
    \hline
    \rowcolor{tgray}
    \multicolumn{3}{|X|}{\textbf{Tabella <nome>}}                                                                                                                                                                                            \\\hline
    \rowcolor{tgray}
    \textbf{Attributo} & \textbf{Tipo di dato} & \textbf{Attributi}\footnote{PK = primary key, NN = not null, UQ = unique, UN = unsigned, AI = auto increment. È ovviamente possibile specificare più di un attributo per ciascuna colonna.} \\\hline
                       &                       &                                                                                                                                                                                             \\ \hline
\end{tabularx}

\subsection*{Strutture di memorizzazione}

\begin{templateblock}
    Compilare la seguente tabella, per ciascuna tabella del database in cui
    sono presenti degli indici.
    Descrivere le motivazioni che hanno portato alla creazione di un indice.
\end{templateblock}

\begin{tabularx}{\linewidth}{|X|X|}
    \hline
    \rowcolor{tgray}
    \multicolumn{2}{|X|}{\textbf{Tabella <nome>}}                                                             \\\hline
    \rowcolor{tgray}
    \textbf{Indice <nome>} & \textbf{Tipo\footnote{IDX = index, UQ = unique, FT = full text, PR = primary.}:} \\\hline
    Colonna 1              & <nome>                                                                           \\ \hline
\end{tabularx}

\subsection*{Trigger}

\begin{templateblock}
    Descrivere quali trigger sono stati implementati, mostrando il codice SQL
    per la loro istanziazione. Si faccia riferimento al fatto che il DBMS di
    riferimento richiede di utilizzare trigger anche per realizzare vincoli di
    check ed asserzioni.
\end{templateblock}

\subsection*{Eventi}

\begin{templateblock}
    Descrivere quali eventi sono stati implementati, mostrando il codice SQL
    per la loro istanziazione. Si descriva anche se gli eventi sono
    istanziati soltanto in fase di configurazione del sistema, o se alcuni
    eventi specifici vengono istanziati in maniera effimera durante
    l’esecuzione di alcune procedure.
\end{templateblock}

\subsection*{Viste}

\begin{templateblock}
    Mostrare e commentare il codice SQL necessario a creare tutte le viste
    necessarie per l’implementazione dell’applicazione.
\end{templateblock}

\subsection*{Stored Procedures e transazioni}

\begin{templateblock}
    Mostrare e commentare le stored procedure che sono state realizzate per
    implementare la logica applicativa delle operazioni sui dati, evidenziando
    quando (e perché) sono state realizzate operazioni transazionali complesse.
\end{templateblock}
