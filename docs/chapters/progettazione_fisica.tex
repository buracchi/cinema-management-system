\section{Progettazione fisica}

\subsection*{Utenti e privilegi}
%
% Descrivere, all’interno dell’applicazione, quali utenti sono stati previsti
% con quali privilegi di accesso su quali tabelle, giustificando le scelte
% progettuali.
%

\subsubsection*{Ruoli}

\begin{tabularx}{\linewidth}{|X|p{12cm}|}
    \hline
    \rowcolor{tblhdrcolor}
    \multicolumn{1}{|c|}{\textbf{Nome}}
                   & \multicolumn{1}{|c|}{\textbf{Privilegi}}
    \\\hline
    Cliente        &
    annulla\_prenotazione(EXECUTE) \linebreak
    elimina\_prenotazione(EXECUTE) \linebreak
    finalizza\_prenotazione(EXECUTE) \linebreak
    inizializza\_prenotazione(EXECUTE) \linebreak
    mostra\_cinema(EXECUTE) \linebreak
    mostra\_palinsesto(EXECUTE) \linebreak
    mostra\_posti\_disponibili(EXECUTE) \linebreak
    mostra\_sale(EXECUTE)
    \\\hline
    Maschera       &
    valida\_prenotazione(EXECUTE)
    \\\hline
    Amministratore &
    assegna\_proiezionista(EXECUTE) \linebreak
    elimina\_cinema(EXECUTE) \linebreak
    elimina\_dipendente(EXECUTE) \linebreak
    elimina\_proiezione(EXECUTE) \linebreak
    elimina\_sala(EXECUTE) \linebreak
    elimina\_turno(EXECUTE) \linebreak
    inserisci\_cinema(EXECUTE) \linebreak
    inserisci\_dipendente(EXECUTE) \linebreak
    inserisci\_proiezione(EXECUTE) \linebreak
    inserisci\_sala(EXECUTE) \linebreak
    inserisci\_turno(EXECUTE) \linebreak
    mostra\_cinema(EXECUTE) \linebreak
    mostra\_cinema\_senza\_maschere(EXECUTE) \linebreak
    mostra\_dipendenti(EXECUTE) \linebreak
    mostra\_film(EXECUTE) \linebreak
    mostra\_proiezioni(EXECUTE) \linebreak
    mostra\_proiezioni\_senza\_proiezionista(EXECUTE) \linebreak
    mostra\_proiezionisti\_disponibili(EXECUTE) \linebreak
    mostra\_sale(EXECUTE) \linebreak
    mostra\_stato\_prenotazioni(EXECUTE) \linebreak
    mostra\_turni(EXECUTE)
    \\\hline
\end{tabularx}

\subsubsection*{Utenti}

\begin{tabularx}{\linewidth}{|X|X|}
    \hline
    \rowcolor{tblhdrcolor}
    \multicolumn{1}{|c|}{\textbf{Nome}}
                   & \multicolumn{1}{|c|}{\textbf{Ruoli}}
    \\\hline
    cliente        & Cliente                              \\\hline
    maschera       & Maschera                             \\\hline
    amministratore & Amministratore                       \\\hline
\end{tabularx}

\pagebreak

\subsection*{Considerazioni sulle strutture di memorizzazione}

È necessario effettuare alcune considerazioni sulla natura del codice di
prenotazione e sulla sua rappresentazione in memoria.

L'iterazione tra un cliente e la sua personale prenotazione avviene attraverso
l'uso del codice di prenotazione, è dunque desiderabile che un utente terzo non
possa interagire con essa intenzionalmente o per errore. Essendo inoltre
il codice utilizzato direttamente dall'utente per interagire con il sistema
ed accedere ai cinema è necessario che questo sia relativamente breve e
disambiguo.

Ricapitolando il codice di prenotazione deve essere univoco, imprevedibile,
breve, stampabile e formato da caratteri facilmente distinguibili.

Poiché la funzione UUID versione 4 definita nell'RFC 4122 genera codici
relativamente lunghi formati da alcuni caratteri difficilmente distinguibili
si è deciso di generare i codici in maniera alternativa.

Si è deciso di utilizzare un codice alfanumerico generato dalla conversione in
base esadecimale di un numero intero generato randomicamente, tale che 
la stringa risultante dalla trasformazione del cambiamento di base rispetti
l'espressione regolare \verb|[1-9A-F]{6}| in modo tale da essere ragionevolmente
breve e formata da caratteri facilmente distinguibili.

Dovendo il codice essere univoco la lunghezza della stringa è stata scelta
tenendo in considerazione il problema delle possibili collisioni durante le
operazioni di inserimento.

Posto $X$ il numero di codici identici generati in $n$ tentativi, in quanto
il fenomeno aleatorio equivale ad uno schema successo insuccesso con rimpiazzo
formato da $n$ prove di Bernoulli indipendenti di parametro
$p = \frac{1}{\left( 9 + 5 \right)^l} = 14^{-l}$,
dove con $l$ indichiamo la lunghezza in caratteri del codice di prenotazione,
possiamo affermare che la variabile aleatoria ha distribuzione binomiale
$X \sim \mathrm{B}\left( n, 14^{-l} \right)$.

Consideriamo ora un periodo temporale della durata di un mese, secondo il volume
atteso dei dati il numero di prenotazioni generato è di un ordine di grandezza
inferiore a $10^6$. Poniamo quindi $n = 10^6$ ed approssimiamo $X$ ad una
variabile aleatoria con distribuzione di Poisson
$\begin{aligned}[t]
        X \sim \mathrm{Poisson}\left( \frac{10^6}{14^l} \right)
    \end{aligned}$,
la probabilità che avvengano dunque collisioni durante un mese è pari a

$\begin{aligned}
        \mathrm{P}\left( X \geq 2 \right)
         & = 1 - \mathrm{P}\left( X = 0 \right)
        - \mathrm{P}\left( X = 1 \right)           \\
         & = 1 - e^{-\frac{10^6}{14^l}}
        - \frac{10^6}{14^l} e^{-\frac{10^6}{14^l}} \\
    \end{aligned}$

In particolare

$\mathrm{P}\left( X \geq 2 \right) \leq 0.5546, \quad$ per $l = 5$

$\mathrm{P}\left( X \geq 2 \right) \leq 0.0081, \quad$ per $l = 6$

\subsection*{Strutture di memorizzazione}
%
% Compilare la tabella seguente indicando quali tipi di dato vengono
% utilizzati per memorizzare le informazioni di interesse nelle tabelle,
% per ciascuna tabella.
%
\begin{tabularx}{\linewidth}{|X|X|X|}
    \hline
    \multicolumn{3}{|>{\columncolor{tblhdrcolor}}l|}{
    \textbf{Cinema}}                               \\\hline
    \rowcolor{tblhdrcolor}
    \multicolumn{1}{|c|}{\textbf{Colonna}}
     & \multicolumn{1}{|c|}{\textbf{Tipo di dato}}
     & \multicolumn{1}{|c|}{\textbf{Attributi}
        \footnote{PK = primary key, NN = not null, UQ = unique,
            UN = unsigned, AI = auto increment.
            È ovviamente possibile specificare più di un attributo per
            ciascuna colonna.}}
    \\\hline
    id
     & INT
     & PK, NN, AI
    \\ \hline
    indirizzo
     & VARCHAR(128)
     & NN, UQ
    \\ \hline
    apertura
     & TIME
     & NN
    \\ \hline
    chiusura
     & TIME
     & NN
    \\ \hline
\end{tabularx}

\begin{tabularx}{\linewidth}{|X|X|X|}
    \hline
    \multicolumn{3}{|>{\columncolor{tblhdrcolor}}l|}{
    \textbf{Sale}}                                 \\\hline
    \rowcolor{tblhdrcolor}
    \multicolumn{1}{|c|}{\textbf{Colonna}}
     & \multicolumn{1}{|c|}{\textbf{Tipo di dato}}
     & \multicolumn{1}{|c|}{\textbf{Attributi}}
    \\\hline
    cinema
     & INT
     & PK, NN
    \\ \hline
    numero
     & INT
     & PK, NN
    \\ \hline
\end{tabularx}

\begin{tabularx}{\linewidth}{|X|X|X|}
    \hline
    \multicolumn{3}{|>{\columncolor{tblhdrcolor}}l|}{
    \textbf{Posti}}                                \\\hline
    \rowcolor{tblhdrcolor}
    \multicolumn{1}{|c|}{\textbf{Colonna}}
     & \multicolumn{1}{|c|}{\textbf{Tipo di dato}}
     & \multicolumn{1}{|c|}{\textbf{Attributi}}
    \\\hline
    cinema
     & INT
     & PK, NN
    \\ \hline
    sala
     & INT
     & PK, NN
    \\ \hline
    fila
     & CHAR(1)
     & PK, NN
    \\ \hline
    numero
     & INT
     & PK, NN
    \\ \hline
\end{tabularx}

\begin{tabularx}{\linewidth}{|X|X|X|}
    \hline
    \multicolumn{3}{|>{\columncolor{tblhdrcolor}}l|}{
    \textbf{Ruoli}}                                \\\hline
    \rowcolor{tblhdrcolor}
    \multicolumn{1}{|c|}{\textbf{Colonna}}
     & \multicolumn{1}{|c|}{\textbf{Tipo di dato}}
     & \multicolumn{1}{|c|}{\textbf{Attributi}}
    \\\hline
    nome
     & VARCHAR(15)
     & PK, NN
    \\ \hline
\end{tabularx}

\begin{tabularx}{\linewidth}{|X|X|X|}
    \hline
    \multicolumn{3}{|>{\columncolor{tblhdrcolor}}l|}{
    \textbf{Dipendenti}}                           \\\hline
    \rowcolor{tblhdrcolor}
    \multicolumn{1}{|c|}{\textbf{Colonna}}
     & \multicolumn{1}{|c|}{\textbf{Tipo di dato}}
     & \multicolumn{1}{|c|}{\textbf{Attributi}}
    \\\hline
    matricola
     & INT
     & PK, NN, AI
    \\ \hline
    nome
     & VARCHAR(45)
     & NN
    \\ \hline
    cognome
     & VARCHAR(45)
     & NN
    \\ \hline
    ruolo
     & VARCHAR(15)
     & NN
    \\ \hline
\end{tabularx}

\begin{tabularx}{\linewidth}{|X|X|X|}
    \hline
    \multicolumn{3}{|>{\columncolor{tblhdrcolor}}l|}{
    \textbf{Giorni}}                               \\\hline
    \rowcolor{tblhdrcolor}
    \multicolumn{1}{|c|}{\textbf{Colonna}}
     & \multicolumn{1}{|c|}{\textbf{Tipo di dato}}
     & \multicolumn{1}{|c|}{\textbf{Attributi}}
    \\\hline
    nome
     & VARCHAR(15)
     & PK, NN
    \\\hline
    numero
     & TINYINT(1)
     & NN, UQ, UN
    \\ \hline
\end{tabularx}

\begin{tabularx}{\linewidth}{|X|X|X|}
    \hline
    \multicolumn{3}{|>{\columncolor{tblhdrcolor}}l|}{
    \textbf{Turni}}                                \\\hline
    \rowcolor{tblhdrcolor}
    \multicolumn{1}{|c|}{\textbf{Colonna}}
     & \multicolumn{1}{|c|}{\textbf{Tipo di dato}}
     & \multicolumn{1}{|c|}{\textbf{Attributi}}
    \\\hline
    dipendente
     & INT
     & PK, NN
    \\ \hline
    giorno
     & VARCHAR(15)
     & PK, NN
    \\ \hline
    inizio
     & TIME
     & PK, NN
    \\ \hline
    durata
     & TIME
     & NN
    \\ \hline
    cinema
     & INT
     & NN
    \\ \hline
\end{tabularx}

\begin{tabularx}{\linewidth}{|X|X|X|}
    \hline
    \multicolumn{3}{|>{\columncolor{tblhdrcolor}}l|}{
    \textbf{Film}}                                 \\\hline
    \rowcolor{tblhdrcolor}
    \multicolumn{1}{|c|}{\textbf{Colonna}}
     & \multicolumn{1}{|c|}{\textbf{Tipo di dato}}
     & \multicolumn{1}{|c|}{\textbf{Attributi}}
    \\\hline
    id
     & INT
     & PK, NN, AI
    \\ \hline
    nome
     & VARCHAR(45)
     & NN
    \\ \hline
    durata
     & TIME
     & NN
    \\ \hline
    casa\_cinematografica
     & VARCHAR(256)
     &
    \\ \hline
    cast
     & VARCHAR(1024)
     &
    \\ \hline
\end{tabularx}

\begin{tabularx}{\linewidth}{|X|X|X|}
    \hline
    \multicolumn{3}{|>{\columncolor{tblhdrcolor}}l|}{
    \textbf{Proiezioni}}                           \\\hline
    \rowcolor{tblhdrcolor}
    \multicolumn{1}{|c|}{\textbf{Colonna}}
     & \multicolumn{1}{|c|}{\textbf{Tipo di dato}}
     & \multicolumn{1}{|c|}{\textbf{Attributi}}
    \\\hline
    cinema
     & INT
     & PK, NN
    \\ \hline
    sala
     & INT
     & PK, NN
    \\ \hline
    data
     & DATE
     & PK, NN
    \\ \hline
    ora
     & TIME
     & PK, NN
    \\ \hline
    prezzo
     & DECIMAL(15,2)
     & NN
    \\ \hline
    film
     & INT
     & NN
    \\ \hline
    proiezionista
     & INT
     &
    \\ \hline
\end{tabularx}

\begin{tabularx}{\linewidth}{|X|X|X|}
    \hline
    \multicolumn{3}{|>{\columncolor{tblhdrcolor}}l|}{
    \textbf{StatiPrenotazioni}}                    \\\hline
    \rowcolor{tblhdrcolor}
    \multicolumn{1}{|c|}{\textbf{Colonna}}
     & \multicolumn{1}{|c|}{\textbf{Tipo di dato}}
     & \multicolumn{1}{|c|}{\textbf{Attributi}}
    \\\hline
    nome
     & VARCHAR(15)
     & PK, NN
    \\ \hline
\end{tabularx}

\begin{tabularx}{\linewidth}{|X|X|X|}
    \hline
    \multicolumn{3}{|>{\columncolor{tblhdrcolor}}l|}{
    \textbf{Prenotazioni}}                         \\\hline
    \rowcolor{tblhdrcolor}
    \multicolumn{1}{|c|}{\textbf{Colonna}}
     & \multicolumn{1}{|c|}{\textbf{Tipo di dato}}
     & \multicolumn{1}{|c|}{\textbf{Attributi}}
    \\\hline
    codice
     & INT
     & PK, NN, AI
    \\ \hline
    transazione
     & VARCHAR(256)
     & UQ
    \\ \hline
    cinema
     & INT
     & NN
    \\ \hline
    sala
     & INT
     & NN
    \\ \hline
    fila
     & CHAR(1)
     & NN
    \\ \hline
    numero
     & INT
     & NN
    \\ \hline
    data
     & DATE
     & NN
    \\ \hline
    ora
     & TIME
     & NN
    \\ \hline
    stato
     & VARCHAR(15)
     & NN
    \\ \hline
    timestamp
     & TIMESTAMP
     & NN
    \\ \hline
\end{tabularx}

\pagebreak

\subsection*{Indici}
%
% Compilare la seguente tabella, per ciascuna tabella del database in cui
% sono presenti degli indici.
% Descrivere le motivazioni che hanno portato alla creazione di un indice.
%
\begin{tabularx}{\linewidth}{|X|X|}
    \hline
    \multicolumn{2}{|>{\columncolor{tblhdrcolor}}l|}{
    \textbf{Proiezioni}}                  \\\hline
    \rowcolor{tblhdrcolor}
    \multicolumn{1}{|l|}{\textbf{idx\_Proiezioni\_Chrono}}
     & \multicolumn{1}{|l|}{\textbf{Tipo}
        \footnote{IDX = index, UQ = unique, FT = full text, PR = primary.}
        \textbf{:}}
    \\\hline
    data, ora, cinema, sala
     & IDX
    \\ \hline
\end{tabularx}

Tramite un'indice mantenente l'ordinamento cronologico delle proiezioni è stato
possibile ottimizzare le operazioni 7, 11 e 22 consentendo di effettuare
range join optimization.

L'indice risulta particolarmente vantaggioso in quanto:
\begin{itemize}
    \item ottimizza l'operazione 11 avente per frequenza attesa il valore
          massimo tra quelli stimati.
    \item l'operazione 8 di inserimento di nuove proiezioni,
          sebbene aggiunga un numero considerevole di proiezioni ogni settimana,
          è un operazione che avviene in blocco in un breve periodo di tempo,
          ammortizzando il costo di aggiornamento degli indici.
    \item l'operazione 9 di cancellazione di una proiezione ha una frequenza
          attesa trascurabile.
    \item l'evento giornaliero di cancellazione delle proiezioni terminate
          avviene in blocco, ammortizzando il costo di aggiornamento degli indici.
\end{itemize}

\pagebreak

\subsection*{Trigger}
%
% Descrivere quali trigger sono stati implementati, mostrando il codice SQL
% per la loro istanziazione. Si faccia riferimento al fatto che il DBMS di
% riferimento richiede di utilizzare trigger anche per realizzare vincoli di
% check ed asserzioni.
%

\inputsqlfile{../server/cinemadb/triggers/Prenotazioni_BEFORE_INSERT_Check_Duplicati.sql}
\inputsqlfile{../server/cinemadb/triggers/Prenotazioni_BEFORE_INSERT_Check_Stato.sql}
\pagebreak
\inputsqlfile{../server/cinemadb/triggers/Prenotazioni_BEFORE_UPDATE_Check_Duplicati.sql}
\inputsqlfile{../server/cinemadb/triggers/Prenotazioni_BEFORE_UPDATE_Check_Ora_Proiezione.sql}
\pagebreak
\inputsqlfile{../server/cinemadb/triggers/Prenotazioni_BEFORE_UPDATE_Check_Stato.sql}
\inputsqlfile{../server/cinemadb/triggers/Prenotazioni_BEFORE_UPDATE_Prevent_Cambio_Codice.sql}
\pagebreak
\inputsqlfile{../server/cinemadb/triggers/Proiezioni_BEFORE_INSERT_Check_Cinema_Aperto.sql}
\inputsqlfile{../server/cinemadb/triggers/Proiezioni_BEFORE_INSERT_Check_Proiezionista.sql}
\pagebreak
\inputsqlfile{../server/cinemadb/triggers/Proiezioni_BEFORE_INSERT_Check_Proiezionista_Occupato.sql}
\pagebreak
\inputsqlfile{../server/cinemadb/triggers/Proiezioni_BEFORE_INSERT_Check_Sala_Libera.sql}
\pagebreak
\inputsqlfile{../server/cinemadb/triggers/Proiezioni_BEFORE_INSERT_Check_Turno.sql}
\inputsqlfile{../server/cinemadb/triggers/Proiezioni_BEFORE_UPDATE_Check_Cinema_Aperto.sql}
\pagebreak
\inputsqlfile{../server/cinemadb/triggers/Proiezioni_BEFORE_UPDATE_Check_Proiezionista.sql}
\inputsqlfile{../server/cinemadb/triggers/Proiezioni_BEFORE_UPDATE_Check_Proiezionista_Occupato.sql}
\pagebreak
\inputsqlfile{../server/cinemadb/triggers/Proiezioni_BEFORE_UPDATE_Check_Sala_Libera.sql}
\pagebreak
\inputsqlfile{../server/cinemadb/triggers/Proiezioni_BEFORE_UPDATE_Check_Turno.sql}
\inputsqlfile{../server/cinemadb/triggers/Turni_AFTER_DELETE_Reset_Proiezionista_Proiezione.sql}
\pagebreak
\inputsqlfile{../server/cinemadb/triggers/Turni_BEFORE_INSERT_Check_Cinema_Aperto.sql}
\inputsqlfile{../server/cinemadb/triggers/Turni_BEFORE_INSERT_Check_Somma_Durate.sql}
\pagebreak
\inputsqlfile{../server/cinemadb/triggers/Turni_BEFORE_INSERT_Check_Sovrapposizioni.sql}
\inputsqlfile{../server/cinemadb/triggers/Turni_BEFORE_UPDATE_Check_Cinema_Aperto.sql}
\pagebreak
\inputsqlfile{../server/cinemadb/triggers/Turni_BEFORE_UPDATE_Check_Somma_Durate.sql}
\inputsqlfile{../server/cinemadb/triggers/Turni_BEFORE_UPDATE_Check_Sovrapposizioni.sql}

\pagebreak

\subsection*{Eventi}
%
% Descrivere quali eventi sono stati implementati, mostrando il codice SQL
% per la loro istanziazione. Si descriva anche se gli eventi sono
% istanziati soltanto in fase di configurazione del sistema, o se alcuni
% eventi specifici vengono istanziati in maniera effimera durante
% l’esecuzione di alcune procedure.
%

Gli eventi sono istanziati unicamente in fase di configurazione del sistema

\inputsqlfile{../server/cinemadb/events/pulizia_prenotazioni.sql}
\inputsqlfile{../server/cinemadb/events/pulizia_prenotazioni_in_attesa.sql}
\inputsqlfile{../server/cinemadb/events/pulizia_proiezioni.sql}
\pagebreak
\inputsqlfile{../server/cinemadb/events/scadenza_prenotazioni.sql}

\subsection*{Viste}
%
% Mostrare e commentare il codice SQL necessario a creare tutte le viste
% necessarie per l’implementazione dell’applicazione.
%

In quanto i requisiti non comprendevano, in senso lato, la predisposizione
del sistema ad interfacciarsi con tecnologie di ORM, non è risultato necessario
implementare alcuna vista.

\pagebreak

\subsection*{Stored Procedures e transazioni}
%
% Mostrare e commentare le stored procedure che sono state realizzate per
% implementare la logica applicativa delle operazioni sui dati, evidenziando
% quando (e perché) sono state realizzate operazioni transazionali complesse.
%

\inputsqlfile{../server/cinemadb/procedures/annulla_prenotazione.sql}
\pagebreak
\inputsqlfile{../server/cinemadb/procedures/assegna_proiezionista.sql}
\pagebreak
\inputsqlfile{../server/cinemadb/procedures/elimina_cinema.sql}
\pagebreak
\inputsqlfile{../server/cinemadb/procedures/elimina_dipendente.sql}
\pagebreak
\inputsqlfile{../server/cinemadb/procedures/elimina_prenotazione.sql}
\pagebreak
\inputsqlfile{../server/cinemadb/procedures/elimina_proiezione.sql}
\pagebreak
\inputsqlfile{../server/cinemadb/procedures/elimina_sala.sql}
\pagebreak
\inputsqlfile{../server/cinemadb/procedures/elimina_turno.sql}
\pagebreak
\inputsqlfile{../server/cinemadb/procedures/finalizza_prenotazione.sql}
\pagebreak
\inputsqlfile{../server/cinemadb/procedures/inizializza_prenotazione.sql}
\pagebreak
\inputsqlfile{../server/cinemadb/procedures/inserisci_cinema.sql}
\inputsqlfile{../server/cinemadb/procedures/inserisci_dipendente.sql}
\inputsqlfile{../server/cinemadb/procedures/inserisci_proiezione.sql}
\pagebreak
\inputsqlfile{../server/cinemadb/procedures/inserisci_sala.sql}
\inputsqlfile{../server/cinemadb/procedures/inserisci_turno.sql}
\pagebreak
\inputsqlfile{../server/cinemadb/procedures/mostra_cinema.sql}
\inputsqlfile{../server/cinemadb/procedures/mostra_cinema_senza_maschere.sql}
\inputsqlfile{../server/cinemadb/procedures/mostra_dipendenti.sql}
\inputsqlfile{../server/cinemadb/procedures/mostra_film.sql}
\pagebreak
\inputsqlfile{../server/cinemadb/procedures/mostra_palinsesto.sql}
\pagebreak
\inputsqlfile{../server/cinemadb/procedures/mostra_posti_disponibili.sql}
\pagebreak
\inputsqlfile{../server/cinemadb/procedures/mostra_proiezioni.sql}
\pagebreak
\inputsqlfile{../server/cinemadb/procedures/mostra_proiezionisti_disponibili.sql}
\pagebreak
\inputsqlfile{../server/cinemadb/procedures/mostra_proiezioni_senza_proiezionista.sql}
\inputsqlfile{../server/cinemadb/procedures/mostra_sale.sql}
\pagebreak
\inputsqlfile{../server/cinemadb/procedures/mostra_stato_prenotazioni.sql}
\inputsqlfile{../server/cinemadb/procedures/mostra_turni.sql}
\pagebreak
\inputsqlfile{../server/cinemadb/procedures/valida_prenotazione.sql}
