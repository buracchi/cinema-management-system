\section{Descrizione del Minimondo}

\begin{tabularx}{\linewidth}{r|>{\internallinenumbers}X|}
    \cline{2-2} &
    %
    % Inserire all’interno di questo riquadro la specifica così come è stata
    % fornita. Nella colonna a sinistra è riportata la numerazione delle righe.
    % Questi numeri dovranno essere utilizzati per riferirsi al testo nelle
    % sezioni successive.
    %
    Si vuole realizzare il sistema informativo di una catena di cinema,
    che si occupi anche della gestione delle prenotazioni.
    \newline
    \newline
    L’amministrazione della catena gestisce i cinema.
    Ciascun cinema ha un numero arbitrario di sale, identificato da un
    numero di sala.
    In ogni sala c’è un numero arbitrario di posti, ciascun individuato
    da una lettera per la fila ed un numero di posto.
    \newline
    \newline
    In ogni sala vengono proiettati più film quotidianamente.
    Ciascun cinema può proiettare lo stesso film più volte in una giornata,
    in sale differenti. Ogni film ha una durata, un nome, è associato
    al cast degli attori protagonisti e una casa cinematografica.
    Lo stesso film, proiettato in orari differenti e in sale differenti,
    può avere un costo per il biglietto differente, proprio in relazione
    alla sala e all’orario in cui esso viene proiettato.
    \newline
    \newline
    Il sistema di prenotazione è tale per cui gli utenti possono prenotarsi
    per una visione, scegliendo un posto disponibile. Dal momento dell’inizio
    della procedura di prenotazione, un cliente ha a disposizione 10 minuti
    per perfezionare la prenotazione.
    Dopo aver scelto il  posto, al cliente è data la possibilità di inserire
    i dati relativi alla propria carta di credito
    (numero, intestatario, data di scadenza, codice CVV).
    Una volta inseriti questi dati, il sistema restituisce all’utente
    un codice di prenotazione.
    Fino a 30 minuti dall'inizio della proiezione, il cliente ha la possibilità
    di annullare la sua prenotazione fornendo al sistema il codice
    di prenotazione.
    \newline
    \newline
    La catena di cinema gestisce anche i propri dipendenti, divisi in
    maschere e proiezionisti.
    Gli amministratori della catena definiscono i turni di lavoro,
    di otto ore massimo. I turni sono gestiti su base settimanale.
    Un report permette di sapere agli amministratori della catena se qualche
    spettacolo è sprovvisto di proiezionista o se qualche cinema, in qualche
    fascia oraria di apertura, è sprovvisto di almeno due maschere per la
    verifica dei biglietti all’ingresso.
    \newline
    \newline
    La verifica dei biglietti avviene da parte delle maschere mediante
    l’utilizzo del codice di prenotazione.
    Un biglietto utilizzato viene associato nel sistema al fatto che quel
    determinato posto, per quella determinata proiezione, è stato occupato
    dallo spettatore.
    Non è possibile utilizzare più volte lo stesso codice di prenotazione
    per accedere al cinema.
    \newline
    \newline
    A fini statistici, gli amministratori possono generare dei report mensili
    che mostrano per ciascun cinema e ciascuna sala quante prenotazioni sono
    state confermate, quante sono state annullate, e quante prenotazioni
    confermate non sono state utilizzate per accedere al cinema.
    \\\cline{2-2}
\end{tabularx}
