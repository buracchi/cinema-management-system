\section{Progettazione concettuale}

\subsection*{Costruzione dello schema E-R}

\begin{templateblock}
    In questa sezione è necessario riportare \underline{tutti} i passi seguiti
    per la costruzione dello schema E-R finale, a partire dalle specifiche
    raccolte ed organizzate nel capitolo precedente.
    Non è richiesto un procedimento specifico: si può adottare una strategia
    top-down, bottom-up, a macchia d’olio o mista.
    L’importante è descrivere e commentare tutti i passi della costruzione,
    andando anche ad inserire “schemi parziali” utilizzati nel processo.
\end{templateblock}

\subsubsection*{Integrazione finale}

\begin{templateblock}
    Nell’integrazione finale delle varie parti dello schema E-R è possibile
    che si evidenzino dei \underline{conflitti sui nomi} utilizzati e dei
    \underline{conflitti strutturali}.
    Prima di riportare lo schema E-R finale, descrivere quali passi sono stati
    adottati per risolvere tali conflitti.
\end{templateblock}

\subsection*{Regole aziendali}

\begin{templateblock}
    Laddove la specifica non sia catturata in maniera completa dallo schema E-R,
    corredare lo stesso in questo paragrafo con l’insieme delle regole
    aziendali necessarie a completare la progettazione concettuale.
\end{templateblock}

\subsection*{Dizionario dei dati}

\begin{templateblock}
    Completare la progettazione concettuale riportando nella tabella seguente
    il dizionario dei dati
\end{templateblock}

\begin{tabularx}{\linewidth}{|l|X|l|l|}
    \hline
    \rowcolor{tblhdrcolor}
    \multicolumn{1}{|c|}{\textbf{Entità}}
     & \multicolumn{1}{|c|}{\textbf{Descrizione}}
     & \multicolumn{1}{|c|}{\textbf{Attributi}}
     & \multicolumn{1}{|c|}{\textbf{Identificatori}}
    \\\hline
    \hfill
     & \hfill
     & \hfill
     & \hfill
    \\ \hline
\end{tabularx}
