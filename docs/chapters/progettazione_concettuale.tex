\section{Progettazione concettuale}

\subsection*{Costruzione dello schema E-R}
%
% In questa sezione è necessario riportare \underline{tutti} i passi seguiti
% per la costruzione dello schema E-R finale, a partire dalle specifiche
% raccolte ed organizzate nel capitolo precedente.
% Non è richiesto un procedimento specifico: si può adottare una strategia
% top-down, bottom-up, a macchia d’olio o mista.
% L’importante è descrivere e commentare tutti i passi della costruzione,
% andando anche ad inserire “schemi parziali” utilizzati nel processo.
%
Per la costruzione dello schema E-R è stata utilizzata una strategia inside-out,
dalla lista dei requisiti si è ricercato il primo concetto idoneo ad essere
rappresento come un'entità, ovvero il cinema, e lo si è modellato provando a
soddisfare i requisiti relativi:

\begin{center}
    \includegraphics[width=5cm]{er_diagrams/er_1.png}
\end{center}

Il concetto successivo è proprio quello di sala, da cui si evince che è
necessario rappresentare anche questo concetto per soddisfare i vincoli
espressi nei requisiti, dunque si procede a reificare l'attributo utilizzando
il pattern "parte-di" in quanto l'esistenza di un'occorrenza dell'entità
sala dipende dall'esistenza di un'occorrenza di un entità cinema:

\begin{center}
    \includegraphics[width=6cm]{er_diagrams/er_2.png}
\end{center}

Il concetto successivo è quello di posto, con un ragionamento analogo al
precedente utilizziamo il pattern "parte-di":

\begin{center}
    \includegraphics[width=11cm]{er_diagrams/er_3.png}
\end{center}

\pagebreak

I concetti successivi sono quelli di proiezione e film. Sebbene non venga
espresso esplicitamente nei requisiti, è importante notare che l'entità
proiezione non può avere come identificatore esterno l'entità film in quanto
non è possibile proiettare film differenti nello stesso momento in una stessa
sala.

Il fatto che l'omonimia sia un'evenienza comune quando si tratta di film
è un'altro aspetto importante da considerare, per risolvere eventuali problemi
introdurremo un codice univoco per identificare l'entità:

\begin{center}
    \includegraphics[width=12cm]{er_diagrams/er_4.png}
\end{center}

\pagebreak

I concetti successivi sono quelli di prenotazione e cliente, le varie tipologie
di prenotazione verranno espresse tramite il costrutto della generalizzazione.

Poiché la cardinalità della relazione che lega l'entità prenotazione con
l'entità proiezione è (1,1) ed il prezzo di una prenotazione è univocamente
identificato dagli attributi che compongono la chiave primaria dell'entità
proiezione, aggiungiamo l'attributo prezzo a quest'ultima.

Notiamo inoltre che un'operazione di rimborso consiste nella richiesta al
servizio di pagamenti elettronici di annullare una transazione, sarà necessario
pertanto registrare il codice identificante la transazione dal servizio
utilizzato mentre non sarà necessario registrare i dati sensibili delle carte
di credito nel sistema.

\begin{center}
    \includegraphics[width=16.5cm]{er_diagrams/er_5.png}
\end{center}

\pagebreak

Gli ultimi concetti da rappresentare sono quelli di dipendente e turno.

\includegraphics[width=\linewidth]{er_diagrams/er_6.png}

Non avendo adottato una strategia bottom-up l'integrazione finale risulta
banale, prima di potervi procedere tuttavia sarà necessario nominare le
relazioni tra le entità nello schema sopra prodotto ed effettuare alcune
considerazioni.

Sebbene non venga esplicitamente richiesto nei requisiti, è fondamentale che
l'entità cinema mantenga informazioni relative al suo indirizzo per poter
essere identificato dal cliente.
È inoltre necessario notare che un indirizzo non è un attributo adatto ad
identificare l'entità in quanto soggetto a possibili variazioni, pertanto
sarà necessario introdurre un codice identificativo artificiale.

Sebbene non venga esplicitamente richiesto nei requisiti, è fondamentale che
l'entità dipendente mantenga alcune informazioni relative alla propria
anagrafica per poter essere identificato dall'amministratore.
È inoltre necessario notare che questi attributi non sono adatti ad
identificare l'entità per problemi di possibili omonimie, pertanto
sarà necessario introdurre un codice identificativo artificiale.

Sebbene non venga esplicitamente richiesto nei requisiti, è fondamentale che
l'entità cliente mantenga informazioni relative alle proprie credenziali per
poter essere identificato dal sistema.
È inoltre necessario notare che questi attributi non sono adatti ad
identificare l'entità, pertanto sarà necessario introdurre un codice
identificativo artificiale.

\subsubsection*{Integrazione finale}
%
% Nell’integrazione finale delle varie parti dello schema E-R è possibile
% che si evidenzino dei \underline{conflitti sui nomi} utilizzati e dei
% \underline{conflitti strutturali}.
% Prima di riportare lo schema E-R finale, descrivere quali passi sono stati
% adottati per risolvere tali conflitti.
%
\includegraphics[width=\linewidth]{er_diagrams/final-integration.png}

\subsection*{Regole aziendali}
%
% Laddove la specifica non sia catturata in maniera completa dallo schema E-R,
% corredare lo stesso in questo paragrafo con l’insieme delle regole
% aziendali necessarie a completare la progettazione concettuale.
%
\begin{longtable}{|p{16.75cm}|}
    \hline
    \rowcolor{tblhdrcolor}
    \multicolumn{1}{|c|}{\textbf{Regole di vincolo}}
    \\\hline
    %\textlangle{}concetto\textrangle{}
    %deve/non deve
    %\textlangle{}espressione\textrangle{} \\
    % Il prezzo di una prenotazione deve dipendere dal film, dalla sala
    % e dal momento in cui avviene la relativa proiezione.        \\\hline
    La procedura di prenotazione deve scadere dopo 10 minuti.   \\\hline
    Le prenotazioni confermate devono poter essere annullate entro 30 minuti
    dall'inizio della proiezione.                               \\\hline
    In una prenotazione devono coincidere il numero di sala e cinema del posto
    e della proiezione.                                         \\\hline
    Una prenotazione deve poter essere validata un'unica volta. \\\hline
    Una prenotazione scaduta non deve poter essere validata.    \\\hline
    Una prenotazione annullata non deve poter essere validata.  \\\hline
    I turni devono essere di massimo otto ore.                  \\\hline
    I turni in un cinema devono avvenire nella fascia oraria d'apertura
    del cinema.                                                 \\\hline
    Le proiezioni in un cinema devono avvenire nella fascia oraria d'apertura
    del cinema.                                                 \\\hline
    Una proiezione deve poter essere associata solamente ad un proiezionista
    libero assegnato ad un turno che copre l'intervallo temporale ed il
    cinema in cui essa avviene.                                 \\\hline
    In una sala non devono poter essere proiettati più film
    contemporaneamente.                                         \\ \hline
    \rowcolor{tblhdrcolor}
    \multicolumn{1}{|c|}{\textbf{Regole di derivazione}}
    \\\hline
    %\textlangle{}concetto\textrangle{}
    %si ottiene
    %\textlangle{}operazione su concetti\textrangle{} \\
    I posti disponibili per una proiezione si ottengono rimuovendo dai posti
    della sala in cui avviene la proiezione i posti associati a tutte le
    prenotazioni confermate relative alla proiezione
    \\\hline
\end{longtable}

\pagebreak

\subsection*{Dizionario dei dati}
%
% Completare la progettazione concettuale riportando nella tabella seguente
% il dizionario dei dati
%
\begin{longtable}{|p{2.5cm}|p{7.1cm}|p{2.5cm}|p{3.2cm}|}
    \hline
    \rowcolor{tblhdrcolor}
    \multicolumn{1}{|c|}{\textbf{Entità}}
     & \multicolumn{1}{|c|}{\textbf{Descrizione}}
     & \multicolumn{1}{|c|}{\textbf{Attributi}}
     & \multicolumn{1}{|c|}{\textbf{Identificatori}}
    \\ \hline
    Cinema
     & Un cinema facente parte della catena cinematografica.
     & Fascia oraria d'apertura,
    indirizzo
     & Id
    \\\hline
    Sala
     & Una sala di un cinema in cui vengono effettuate proiezioni.
     &
     & Numero, Cinema
    \\\hline
    Posto
     & Sedia, poltrona e sim. in una sala prenotabile da un cliente per
    una proiezione.
     &
     & Fila, Numero, Sala
    \\\hline
    Proiezione
     & Riproduzione di un film in una sala in una certa data e in un
    certo orario.
     & Prezzo
     & Data, Orario, Sala
    \\\hline
    Film
     & Produzione cinematografica proiettabile in una sala di un cinema.
     & Casa cinematografica, Nome, Durata, Cast
     & Id
    \\\hline
    Prenotazione
     & Diritto a sedere su un certo posto per una certa proiezione.
     & Codice transazione
     & Codice di prenotazione
    \\\hline
    Prenotazione annullata
     & Una prenotazione che è stata annullata dal cliente.
     &
     &
    \\\hline
    Prenotazione confermata
     & Prenotazione non annullata.
     &
     &
    \\\hline
    Prenotazione validata
     & Prenotazione confermata validata da una maschera.
     &
     &
    \\\hline
    Prenotazione scaduta
     & Una prenotazione confermata relativa ad proiezione effettuata non
    validata da una maschera.
     &
     &
    \\\hline
    Dipendente
     & Un dipendente della catena cinematografica.
     & Nome,

    Cognome
     & Matricola
    \\\hline
    Maschera
     & Un tipo di dipendente.
     &
     &
    \\\hline
    Proiezionista
     & Un tipo di dipendente.
     &
     &
    \\\hline
    Turno
     & Un’intervallo di tempo di massimo otto ore all’interno di una giornata
    che definisce il periodo lavorativo di un dipendete.
     &
     & Giorno della settimana, Orario
    \\\hline
    Cliente
     & Un utente che può effettuare la procedura di prenotazione.
     & Credenziali
     & Id
    \\\hline
\end{longtable}

%\begin{longtable}{|p{2.5cm}|p{7.6cm}|p{3.2cm}|p{2cm}|}
%    \hline
%    \rowcolor{tblhdrcolor}
%    \multicolumn{1}{|c|}{\textbf{Relazione}}
%     & \multicolumn{1}{|c|}{\textbf{Descrizione}}
%     & \multicolumn{1}{|c|}{\textbf{Entità coinvolte}}
%     & \multicolumn{1}{|c|}{\textbf{Attributi}}
%    \\\hline
%    \hfill
%     & \hfill
%     & \hfill
%     & \hfill
%    \\ \hline
%\end{longtable}
